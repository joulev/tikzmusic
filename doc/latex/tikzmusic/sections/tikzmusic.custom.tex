\section{Customization}\label{sec:custom}
\subsection{Color}\label{sec:custom:color}
\begin{key}{/tm/line color=\meta{color} (initially black)}
  Set the color of the lines, including the five main lines in each staff and 
  the additional lines in the notes.
\end{key}
\begin{key}{/tm/color=\meta{color} (initially black)}
  Set the color of everything except the lines, whose color has been handled using 
  |line color|.
\end{key}
\begin{codeexample}[]
\begin{tmline}
\begin{tmstaff}[color=red,line color=blue]{g}{}
  \tmquarter[color=teal,line color=purple]{8}{G3}{}
\end{tmstaff}
\end{tmline}
\end{codeexample}
You can specify colors in a more compact way. If you want to set |color| and 
|line color| to the same color \meta{color}, you can use \meta{color} as an 
option. This works pretty much like the way you use colors in \tikzname. However, 
keep in mind that \emph{both} |line color| and |color| are affected by specifing 
this way.
\begin{codeexample}[]
\begin{tmline}
\begin{tmstaff}{g}{}
  \tmquarter[red]{6}{G3}{}
  \tmquarter[color=red]{8}{G3}{}
\end{tmstaff}
\end{tmline}
\end{codeexample}
When adding notations to a note, e.g. when you use |\tmappendaccidental| to a 
note, you might want to set the color of the additional notation to be the same 
as the color of that note. You can do so using the following key:
\begin{key}{/tm/use note color=\meta{|true| or |false|} (default true)}
  If this key is set to |true|, the color of the additional notation is set to 
  the color of its `parent' note.
\end{key}
\begin{codeexample}[]
\begin{tmline}
\begin{tmstaff}{g}{}
  \tmquarter[color=red]{8}{G3}{a}\tmappendaccidental[use note color]{a}{G3}{sharp}
\end{tmstaff}
\end{tmline}
\end{codeexample}
\subsection{Note length}\label{sec:custom:note-length}
\begin{key}{/tm/note length=\meta{length} (initially 6mm)}
  Reset the stem length of notes.
\end{key}
\begin{codeexample}[]
\begin{tmline}
\begin{tmstaff}{g}{}
  \begin{tmbeam}[note length=1.7cm]
    \tmbeamnote{4}{C4}{10}{a}\tmbeamnote{5}{D4}{10}{b}\tmbeamnote{6}{E4}{10}{c}
    \tmbeamnote{7}{F4}{10}{d}\tmbeamnote{8}{G4}{10}{e}\tmbeamnote{9}{A4}{10}{f}
  \end{tmbeam}
  \tmquarter[reversed]{10}{B4}{}
\end{tmstaff}
\end{tmline}
\end{codeexample}
\subsection{Reversing}\label{sec:custom:reverse}
\begin{key}{/tm/reversed=\meta{|true| or |false|} (default |true|)}
  This key can alter the way some notations look:

  \begin{itemize}
    \item If this is applied to |\tmhalf|, |\tmquarter| and friends, this will 
    change the direction of the stem. Note that this key does nothing when being 
    applied to a beamed environment (see more in sectioin \ref{sec:music-notes:commands}).
    \item If this key is applied to |\tmslur|, the direction of the 
    slur will be reversed (see more in section \ref{sec:line:slur}).
    \item If this key is applied to |\tmtuplets|, it will change the position of 
    the tuplet in relative to the notes (see more in section \ref{sec:line:tuplet}).
  \end{itemize}
\end{key}
\begin{codeexample}[]
\begin{tmline}
\begin{tmstaff}{g}{}
  \tmquarter{3}{C5}{a}\tmquarter[reversed]{4}{C5}{b}
  \tmquarter{5}{C5}{c}\tmquarter{6}{C5}{d}\tmslur[reversed]{c}{d}\tmtuplets[reversed]{c}{d}{2}
\end{tmstaff}
\end{tmline}
\end{codeexample}
\subsection{Transformations}\label{sec:custom:transformations}
\subsubsection{Shifting for lines}\label{sec:custom:transformations:lines}
These applies to the lines command, see section \ref{sec:line} and |\tmbrace| and 
|\tmbracket|.
\begin{key}{/tm/start xshift=\meta{length} (initially 0pt)}
  Shift the starting point of the line by \meta{length} in the horizontal direction.
\end{key}
\begin{codeexample}[]
\begin{tmline}
\begin{tmstaff}{g}{}
  \tmquarter{5}{C4}{a}\tmquarter{8}{C5}{b}
  \tmoctave[red,start xshift=-5mm]{a}{b}{8va}
  \tmoctave{a}{b}{8va} % For comparison
\end{tmstaff}
\end{tmline}
\end{codeexample}
\begin{key}{/tm/start yshift=\meta{length} (initially 0pt)}
  Shift the starting point of the line by \meta{length} in the vertical direction.
\end{key}
\begin{key}{/tm/start shift=\meta{coordinate}}
  Shift the starting point of the line by \meta{coordinate}.
\end{key}
\begin{key}{/tm/end xshift=\meta{length} (initially 0pt)}
  Shift the ending point of the line by \meta{length} in the horizontal direction.
\end{key}
\begin{key}{/tm/end yshift=\meta{length} (initially 0pt)}
  Shift the ending point of the line by \meta{length} in the vertical direction.
\end{key}
\begin{key}{/tm/end shift=\meta{coordinate}}
  Shift the ending point of the line by \meta{coordinate}.
\end{key}
\subsubsection{Brace-specific shifting}\label{sec:custom:transformations:brace}
\begin{key}{/tm/brace middle xshift=\meta{length} (initially 0pt)}
  Shift the middle point of the brace by \meta{length} in the horizontal direction.
\end{key}
\begin{codeexample}[]
\begin{tmline}[staff offset=1.7cm]
\begin{tmstaff}{g}{a}\end{tmstaff}%
\begin{tmstaff}{g}{b}\end{tmstaff}%
\tmbrace[brace middle xshift=-3mm]{a}{b}{Piano}%
\end{tmline}
\end{codeexample}
\begin{key}{/tm/brace middle yshift=\meta{length} (initially 0pt)}
  Shift the middle point of the brace by \meta{length} in the vertical direction.
\end{key}
\begin{key}{/tm/brace middle shift=\meta{coordinate}}
  Shift the middle point of the brace by \meta{coordinate}.
\end{key}
\subsubsection{Shifting for others}\label{sec:custom:transformations:others}
Keep in mind that shifting only works with some commands, like the articulations 
or the accidentals. Things whose coordinates are already specified, e.g. 
|\tmwhole|, may or may not be affected by these keys.
\begin{key}{/tm/xshift=\meta{length} (initially 0pt)}
  Shift the object by \meta{length} in the horizontal direction. In the lines 
  this is a shorthand for |start xshift| and |end xshift|.
\end{key}
\begin{codeexample}[]
\begin{tmline}
\begin{tmstaff}{g}{}
  \tmquarter{6}{E4,B4}{a}
  \tmappendaccidental{a}{E4}{flat}\tmappendaccidental[xshift=-2mm]{a}{B4}{flat}
\end{tmstaff}
\end{tmline}
\end{codeexample}
\begin{key}{/tm/yshift=\meta{length} (initially 0pt)}
  Shift the object by \meta{length} in the horizontal direction. In the lines 
  this is a shorthand for |start yshift| and |end yshift|
\end{key}
\begin{key}{/tm/shift=\meta{coordinate}}
  Shift the object by \meta{coordinate} in the horizontal direction. In the lines 
  this is a shorthand for |start shift| and |end shift|.
\end{key}
\begin{key}{/tm/line shift=\meta{number} (initially 0)}
  Shift the object by \meta{number} `note' lines. For example, if an accidental 
  is displayed for a |D4| note, |line shift=1| will make it being displayed for 
  a (possibly imaginary) |E4| (\emph{not} |F4|!) note at that position. This is 
  effectively |yshift| where \meta{length} is set to \meta{number} $\times$ 
  \meta{half of line sep}.
\end{key}
See the following example for more information about how |line shift| works:
\begin{codeexample}[]
\begin{tmline}
\begin{tmstaff*}{}
  \tmeighthrest[line shift=4]{3}\tmeighthrest[line shift=-4]{3}
  \tmwholerest[line shift=4]{6} \tmwholerest[line shift=-4]{6}
\end{tmstaff*}
\end{tmline}
\end{codeexample}
