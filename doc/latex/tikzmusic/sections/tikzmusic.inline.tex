\section{Other in-line stuffs}\label{sec:inline}
\subsection{Clefs}\label{sec:inline:clef}
You can add a clef in-line using the following commands:
\begin{command}{\tmgclef\opt{\oarg{options}}\marg{x-pos}}
  Add a treble clef at position \meta{x-pos}. The clef will be scaled down a bit 
  as per standards. \meta{shift} works like in |\tmkeysignature|.
\end{command}
\begin{command}{\tmcclef\opt{\oarg{options}}\marg{x-pos}}
  Work like |\tmgclef|, but the clef is the alto clef.
\end{command}
\begin{command}{\tmfclef\opt{\oarg{options}}\marg{x-pos}}
  Work like |\tmgclef|, but the clef is the bass clef.
\end{command}
\begin{codeexample}[]
\begin{tmline}
\begin{tmstaff}{g}{}
  \tmquarter{3}{C4}{}
  \tmfclef{4}\tmquarter{5}{C4}{}
  \tmcclef{6}\tmquarter{7}{C4}{}
  \tmgclef{8}\tmquarter{9}{C4}{}
  \tmcclef[line shift=2]{10}\tmquarter{11}{C4}{}
\end{tmstaff}
\end{tmline}
\end{codeexample}
However, sometimes you don't want these clefs to be scaled. In those cases, you 
can use the following key:
\begin{key}{/tm/unscaled=\meta{|true| or |false|} (default true)}
  Unscale the staves drawn by |\tmgclef| and friends.
\end{key}
\begin{codeexample}[]
\begin{tmline}
\begin{tmstaff}[unscaled]{g}{}
  \tmfclef{4}\tmcclef{6}\tmgclef{8}\tmcclef[line shift=2]{10}
\end{tmstaff}%
\end{tmline}
\end{codeexample}
\subsection{Breaths}\label{sec:inline:breaths}
\begin{command}{\tmbreath\opt{\oarg{options}}\marg{x-pos}}
  Add a breath mark (a comma) to position \meta{x-pos}.
\end{command}
\begin{command}{\tmcaesura\opt{\oarg{options}}\marg{x-pos}}
  Add a caesura to position \meta{x-pos}.
\end{command}
\begin{codeexample}[]
\begin{tmline}
\begin{tmstaff}{g}{}
  \tmbreath{4}\tmbreath[line shift=1]{5}\tmcaesura{8}\tmcaesura[line shift=-4]{9}
\end{tmstaff}
\end{tmline}
\end{codeexample}
\subsection{Dynamics}\label{sec:inline:dynamics}
\begin{command}{\tmdynamics\opt{\oarg{options}}\marg{coordinate}\marg{type}}
  Add a dynamics notation to \meta{coordinate}. \meta{type} can be one of the 
  following values: |mp|, |p|, |pp|, |ppp|, |mf|, |f|, |ff|, |fff| and |fp|.
\end{command}
\begin{codeexample}[]
\begin{tmline}
\begin{tmstaff}{g}{}
  \tmdynamics{4,-1}{pp}\tmdynamics[red]{8,1}{mf}
\end{tmstaff}
\end{tmline}
\end{codeexample}
\subsection{Arpeggios and glissandi}\label{sec:inline:arpeggio-glissando}
\subsubsection{Arpeggios}\label{sec:inline:arpeggio-glissando:arpeggio}
\begin{command}{\tmarpeggio\opt{\oarg{options}}\marg{note}}
  Draw an arpeggio to \meta{note}. Note that the starting point (for |start xshift| 
  usage...) is the lower point.
\end{command}
\begin{command}{\tmarpeggioup\opt{\oarg{options}}\marg{note}}
  Draw an `increasiing' arpeggio to \meta{note}.
\end{command}
\begin{command}{\tmarpeggiodown\opt{\oarg{options}}\marg{note}}
  Draw a `decreasing' arpeggio to \meta{note}.
\end{command}
\begin{codeexample}[]
\begin{tmline}
\begin{tmstaff}{g}{}
  \tmquarter{4}{C4,E4,G4,C5}{a}     \tmarpeggio{a}
  \tmquarter{6}{C4,E4,G4,C5}{a}     \tmarpeggioup[start xshift=-2mm]{a}
  \tmquarter[red]{8}{C4,E4,G4,C5}{a}\tmarpeggiodown[use note color]{a}
\end{tmstaff}
\end{tmline}
\end{codeexample}
\subsubsection{Glissandi}\label{sec:inline:arpeggio-glissando:glissando}
\begin{command}{\tmglissando\opt{\oarg{options}}\marg{note 1}\marg{note 2}}
  Draw a glissando from the lowest note of \meta{note 1} to the highest note of 
  \meta{note 2}.
\end{command}
\begin{codeexample}[]
\begin{tmline}
\begin{tmstaff}{g}{}
  \tmquarter{4}{C3}{a}\tmquarter{7}{C6}{b}\tmglissando{a}{b}
\end{tmstaff}
\end{tmline}
\end{codeexample}
By default, the word \emph{gliss.} will be displayed. You can hide that word 
(e.g. when the two notes are too close) by using this key:
\begin{key}{/tm/hide text=\meta{|true| or |false|} (default true)}
  Hide the word \emph{gliss.} in glissando commands.
\end{key}
\begin{codeexample}[]
\begin{tmline}
\begin{tmstaff}{g}{}
  \tmquarter{4}{C3}{a}\tmquarter{7}{C6}{b}\tmglissando[hide text]{a}{b}
\end{tmstaff}
\end{tmline}
\end{codeexample}
\begin{command}{\tmglissandocoordinate\opt{\oarg{options}}\marg{coordinate 1}\marg{coordinate 2}}
  Draw a `glissando line' from \meta{coordinate 1} to \meta{coordinate 2}.
\end{command}
\begin{codeexample}[]
\begin{tmline}
\begin{tmstaff}{g}{}
  \tmglissandocoordinate{3,.6}{6,-.6}\tmglissandocoordinate[hide text]{7,.6}{10,-.6}
\end{tmstaff}
\end{tmline}
\end{codeexample}
\subsection{Repeat notations}\label{sec:inline:repeats}
\begin{command}{\tmmeasurerepeat\opt{\oarg{options}}\marg{x-pos}}
  Add a `measure-repeat' notation (\tikz\pic{tm-measure-repeat};) at \meta{x-pos}.
\end{command}
\begin{command}{\tmsegno\opt{\oarg{options}}\marg{x-pos}}
  Add a segno at \meta{x-pos}.
\end{command}
\begin{command}{\tmcoda\opt{\oarg{options}}\marg{x-pos}}
  Add a coda at \meta{x-pos}.
\end{command}
\begin{codeexample}[]
\begin{tmline}
\begin{tmstaff}{g}{}
  \tmmeasurerepeat{3}\tmsegno{5}\tmsegno[yshift=5mm]{7}\tmcoda{9}\tmcoda[red]{11}
\end{tmstaff}
\end{tmline}
\end{codeexample}
