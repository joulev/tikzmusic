\section{Initialization}\label{sec:init}
\subsection{Loading the package}\label{sec:init:load}
This package currently only supports \LaTeXe.

\begin{package}{tikzmusic}
  Loading the \tmname\ package. There are no package options.
\end{package}

This package will automatically load the packages \pkg{spath3} 
and \tikzname, as well as \tikzname\ standard libraries |calc|, 
|intersection|, |decorations.pathreplacing|. You don't need to load these 
packages and libraries again in your document.
\subsection{Processing options}\label{sec:init:options}
The \tmname\ package uses the \pkg{pgfkeys} package to handle options. Every 
option defined in the package is in the same family, |/tm|, e.g. 
|color|.

\begin{command}{\tmset\marg{options}}
  Process \meta{options}. where the default path is set to |/tm|.
\end{command}

If you know about \tikzname\ and its key system, you can think |\tmset| 
works just like |\tikzset|, only the default path is different. You can now skip 
to the next section.

If you are not familiar with \pkg{pgfkeys} or |\pgfkeys| or 
|\tikzset|, \meta{options} is a list of \meta{key}\opt{|=|\meta{value}} 
pairs, separated by commas. The command will then separate each pair and process 
them.

\begin{itemize}
  \item If the key is with \meta{value}, option \meta{key}
  is processed, with its value being \meta{value}.
  \item Otherwise, the command will check whether \meta{key} is a defined key. 
  If it is defined, option \meta{key} is processed. 
  Otherwise, it will be processed as the value for both |line color| and |color| 
  (see section \ref{sec:custom:color}).
\end{itemize}

If you want to learn deeper about this, you can read section 88 of the PGF 
manual (you can read it via |texdoc pgf|).
\subsection{Environments for music lines}\label{sec:init:drawing-environment}
Each music line will be drawn separately, by using the following environment:

\begin{environment}{{tmline}\opt{\oarg{options}}}
  Add a music line (consisting of one or many staves).
\end{environment}

\begin{codeexample}[]
\begin{tmline}[blue]
\begin{tmstaff}{g}{}\end{tmstaff}
\end{tmline}
\end{codeexample}

If a line consists of more than one staff, you may need to indent the staves a 
little bit to make room for instrument names and braces/brackets. You can do so 
by using the following key:

\begin{key}{/tm/staff offset=\meta{length} (initially 0pt)}
  Indent all staves in a line by \meta{length}.
\end{key}

\begin{codeexample}[]
\begin{tmline}[staff offset=1.5cm]
\begin{tmstaff}{g}{a}
  \path[overlay] (a-start) node[left] {Violin};%
\end{tmstaff}
\end{tmline}%
\end{codeexample}
\subsection{Creating a staff}\label{sec:init:staff-creation}
A staff can be created using one of the following environments:
\begin{environment}{{tmstaff}\opt{\oarg{options}}\marg{clef name}\marg{staff name}}
  Create a staff, with the starting clef is \meta{clef name}.

  \meta{clef name} can have three values: |g|, |f| and |c|, which stands for 
  the treble (G) clef, the bass (F) clef and the alto (C) clef, respectively.

  \meta{staff name} will be used to make cross-staff barlines or braces, so 
  you should only left it empty if you are sure you will not refer to it later.

  \meta{options} will be executed at the beginning of the environment.
\end{environment}
\begin{codeexample}[]
\begin{tmline}[staff offset=1cm]
\begin{tmstaff}{g}{}\end{tmstaff}
\end{tmline}
\end{codeexample}
\begin{environment}{{tmstaff*}\opt{\oarg{options}}\marg{staff name}}
  Work like |{tmstaff}|, but no clefs will be drawn.
\end{environment}
Essentially, |{tmstaff}| and |{tmstaff*}| are extensions of the 
|{tikzpicture}| environment, where the origin of the canva is the leftmost 
point of the middle line. That origin is marked as \tikzname\ remembered 
coordinate |(|\meta{staff name}|-start)|. There are also two other remembered 
coordinates: the leftmost points of the top line and the bottom line are marked 
as \tikzname\ coordinates |(|\meta{staff name}|-nw)| and |(|\meta{staff name}|-sw)| 
respectively.
\begin{codeexample}[]
\begin{tmline}
\begin{tmstaff}{g}{my-staff}
  \path (my-staff-nw) node[above,overlay] {1} ++ (.5,.5) node[right] {\bfseries Allegro};
\end{tmstaff}
\end{tmline}
\end{codeexample}
