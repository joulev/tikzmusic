\section{Lines}\label{sec:line}
\subsection{Slur}\label{sec:line:slur}
\begin{command}{\tmslur\opt{\oarg{options}}\marg{note 1}\oarg{shift 2}\marg{note 2}}
  Draw a slur joining \meta{note 1} and \meta{note 2}. The slur will join the 
  \emph{lowest} notes of the two note set, i.e. it will go down and then go 
  up.\footnote{Not being a native speaker, I can't find an 
  appropriate English word for this :).} 
  You can change this direction using |reversed|.
\end{command}
\begin{codeexample}[]
\begin{tmline}
\begin{tmstaff}{g}{}
  \tmquarter{3}{C4}{a}\tmquarter{5}{D4}{b}\tmslur{a}{b}
  \tmquarter{7}{C5}{a}\tmquarter{9}{D5}{b}
  \tmslur[reversed,amplitude=1.5mm,start yshift=-1mm,end shift={(-1mm,-1mm)}]{a}{b}
\end{tmstaff}
\end{tmline}
\end{codeexample}
\begin{command}{\tmslurcoordinate\opt{\oarg{options}}\marg{coordinate 1}\marg{coordinate 2}}
  Draw a slur from \meta{coordinate 1} to \meta{coordinate 2}. The slur will go 
  down and then go up.
\end{command}
You can use this command to tie two notes as follows. It is how |\tmtie| is
currently working.
\begin{codeexample}[]
\begin{tmline}
\begin{tmstaff}{g}{}
  \tmquarter{4}{C4,E4,G4}{a}\tmquarter{7}{C4,E4,G4}{b}
  \tmslurcoordinate[amplitude=1.5mm,start shift={(2mm,-1.5mm)},end shift={(-2mm,-1.5mm)}]{a-C4}{b-C4}
  \tmslurcoordinate[amplitude=1.5mm,start shift={(2mm,-1.5mm)},end shift={(-2mm,-1.5mm)}]{a-E4}{b-E4}
  \tmslurcoordinate[reversed,amplitude=1.5mm,start shift={(2mm,1.5mm)},end shift={(-2mm,1.5mm)}]{a-G4}{b-G4}
\end{tmstaff}
\end{tmline}
\end{codeexample}
Essentially, the slur is drawn using the |calligraphic curved parenthesis| 
decoration, offered by the \pkg{spath3} package. You can control the amplitude of 
this decoration, a.k.a. the height of the slur, by the following key:
\begin{key}{/tm/amplitude=\meta{length} (initially 2.5mm)}
  Control the amplitude of the slurs.
\end{key}
\begin{codeexample}[]
\begin{tmline}
\begin{tmstaff}{g}{}
  \tmquarter{6}{E4}{a}\tmquarter{9}{F4}{b}
  \tmslur{a}{b}\tmslur[red,amplitude=4mm]{a}{b}
\end{tmstaff}
\end{tmline}
\end{codeexample}
\subsection{Tying notes}\label{sec:line:tie}
\begin{command}{\tmtie\opt{\oarg{options}}\marg{note 1}\marg{note 2}}
  Add a tie between \meta{note 1} and \meta{note 2}.
\end{command}
\begin{codeexample}[]
\begin{tmline}
\begin{tmstaff}{g}{}
  \tmquarter{4}{C4,E4,G4}{a}\tmquarter{6}{C4,E4}{b}\tmtie{b}{a}
  %\tmtie{a}{b}: Error
\end{tmstaff}
\end{tmline}
\end{codeexample}
{\bfseries\sffamily Important note:}
\begin{itemize}
  \item |\tmtie| is intended to be used with note sets having more than one notes. 
  Of course, with note sets having just one note it still works, but expected 
  behaviour is not guaranteed. In those cases, use |\tmslur| and friends, 
  documented in section \ref{sec:line:slur}, instead.
  \item The number of notes in \meta{note 1} \emph{must not} be more than that in 
  \meta{note 2}. So, the order matters -- in the example above, you can't use 
  |\tmtie{a}{b}| because that will resulted in error. Of course, if \meta{note 1} 
  and \meta{note 2} have the same number of notes, which is very usually the case, 
  you can use any order as you want.
  \item Note that the starting coordinate is always the one having the less 
  $x$-coordinate, no matter how the notes are ordered in |\tmtie|. In the example 
  above, the starting coordinate is |a|, although it comes after |b|. So 
  |start xshift| (say) will be applied to |a|, not |b|.
\end{itemize}
\begin{codeexample}[]
\begin{tmline}
\begin{tmstaff}{g}{}
  \tmquarter{4}{C4,E4,G4}{a}\tmquarter{6}{C4,E4}{b}\tmtie[start xshift=-1cm]{b}{a}
\end{tmstaff}
\end{tmline}
\end{codeexample}
\subsection{Crescendo and diminuendo}\label{sec:line:cresc-dim}
\subsubsection{Crescendo}\label{sec:line:cresc-dim:cresc}
\begin{command}{\tmcrescendohairpin\opt{\oarg{options}}\marg{note 1}\marg{note 2}}
  Draw a crescendo hairpin between \meta{note 1} and \meta{note 2}.
\end{command}
\begin{codeexample}[]
\begin{tmline}
\begin{tmstaff}{g}{}
  \tmquarter{2}{C5}{a} \tmquarter{4}{D5}{b} \tmcrescendohairpin{a}{b}
  \tmquarter{6}{C5}{a} \tmquarter{8}{D5}{b} \tmcrescendohairpin[yshift=-5mm]{a}{b}
  \tmquarter{10}{C5}{a}\tmquarter{12}{D5}{b}\tmcrescendohairpin[cresc dim sep=5mm]{a}{b}
\end{tmstaff}
\end{tmline}
\end{codeexample}
\begin{command}{\tmcrescendoline\opt{\oarg{options}}\marg{note 1}\marg{note 2}}
  Draw a crescendo line between \meta{note 1} and \meta{note 2}.
\end{command}
\begin{codeexample}[]
\begin{tmline}
  \begin{tmstaff}{g}{}
    \tmquarter{2}{C5}{a}\tmquarter{4}{D5}{b}\tmcrescendoline{a}{b}
    \tmquarter{6}{C5}{a}\tmquarter{8}{D5}{b}\tmcrescendoline[yshift=-5mm]{a}{b}
  \end{tmstaff}
\end{tmline}
\end{codeexample}
\begin{command}{\tmcrescendo\opt{\oarg{options}}\marg{note 1}\marg{note 2}}
  Alias of |\tmcrescendohairpin|.
\end{command}
\begin{command}{\tmcrescendohairpincoordinate\opt{\oarg{options}}\marg{coordinate 1}\marg{coordinate 2}}
  Draw a crescendo hairpin between \meta{coordinate 1} and \meta{coordinate 2}. 
  The coordinates do \emph{not} include parentheses.
\end{command}
\begin{command}{\tmcrescendolinecoordinate\opt{\oarg{options}}\marg{coordinate 1}\marg{coordinate 2}}
  Draw a crescendo line between \meta{coordinate 1} and \meta{coordinate 2}.
\end{command}
\begin{codeexample}[]
\begin{tmline}
\begin{tmstaff}{g}{}
  \tmcrescendohairpincoordinate{1.5,-1}{\linewidth-2mm,-1}
  \tmcrescendolinecoordinate{1.5,-1.5}{\linewidth-2mm,-1.5}
\end{tmstaff}
\end{tmline}
\end{codeexample}
\begin{command}{\tmcrescendocoordinate\opt{\oarg{options}}\marg{coordinate 1}\marg{coordinate 2}}
  Alias of |\tmcrescendohairpincoordinate|.
\end{command}
\subsubsection{Diminuendo}\label{sec:line:cresc-dim:dim}
All commands are just like in crescendo (section \ref{sec:line:cresc-dim:cresc}).
\begin{command}{\tmdiminuendohairpin\opt{\oarg{options}}\marg{note 1}\marg{note 2}}
  Add a diminuendo hairpin between \meta{note 1} and \meta{note 2}.
\end{command}
\begin{command}{\tmdiminuendoline\opt{\oarg{options}}\marg{note 1}\marg{note 2}}
  Add a diminuendo line between \meta{note 1} and \meta{note 2}.
\end{command}
\begin{command}{\tmdiminuendo\opt{\oarg{options}}\marg{note 1}\marg{note 2}}
  Alias of |\tmdiminuendohairpin|.
\end{command}
\begin{command}{\tmdiminuendohairpincoordinate\opt{\oarg{options}}\marg{coordinate 1}\marg{coordinate 2}}
  Add a diminuendo hairpin between \meta{coordinate 1} and \meta{coordinate 2}.
\end{command}
\begin{command}{\tmdiminuendolinecoordinate\opt{\oarg{options}}\marg{coordinate 1}\marg{coordinate 2}}
  Add a diminuendo line between \meta{coordinate 1} and \meta{coordinate 2}.
\end{command}
\begin{command}{\tmdiminuendocoordinate\opt{\oarg{options}}\marg{coordinate 1}\marg{coordinate 2}}
  Alias of |\tmdiminuendohairpincoordinate|.
\end{command}
\begin{codeexample}[]
\begin{tmline}
\begin{tmstaff}{g}{}
  \tmquarter{2}{C4}{a}\tmquarter{4}{G3}{b}\tmdiminuendo{a}{b}
  \tmquarter{5}{C4}{a}\tmquarter{7}{G3}{b}\tmdiminuendoline{a}{b}
  \tmdiminuendocoordinate{8,-1}{10,-1}\tmdiminuendolinecoordinate{11,-1}{13,-1}
\end{tmstaff}
\end{tmline}
\end{codeexample}
\subsubsection{Customization}\label{sec:line:cresc-dim:custom}
You can change the head-width of the crescendo/diminuendo hairpins using the 
following key:
\begin{key}{/tm/cresc dim sep=\meta{length} (initially 3mm)}
  Set the width of the head of the crescendo/diminuendo hairpins.
\end{key}
\begin{codeexample}[]
\begin{tmline}
\begin{tmstaff}{g}{}
  \tmquarter{3}{B4}{a}\tmquarter{6}{C5}{b}\tmdiminuendo{a}{b}
  \tmquarter{9}{B4}{a}\tmquarter{12}{C5}{b}\tmdiminuendo[cresc dim sep=5mm]{a}{b}
\end{tmstaff}
\end{tmline}
\end{codeexample}
\subsection{Volta}\label{sec:line:volta}
\begin{command}{\tmvolta\opt{\oarg{options}}\marg{note 1}\marg{note 2}\marg{number}}
  Draw a volta line between \meta{note 1} and \meta{note 2}.
\end{command}
By default, volta lines are closed. You can `unclose' it by using
\begin{key}{/tm/volta unclosed=\meta{|true| or |false|} (default true)}
  If set to |true|, the volta will be unclosed.
\end{key}
\begin{codeexample}[]
\begin{tmline}
\begin{tmstaff}{g}{}
  \tmbarlineinline{4}
  \tmquarter{5}{C4}{a}\tmquarter{6}{D4}{}\tmquarter{7}{E4}{b}
  \tmendrepeatbarlineinline{8}
  \tmquarter{9}{F4}{c}\tmquarter{10}{G4}{}\tmquarter{11}{A4}{d}
  \tmvolta{a}{b}{1}\tmvolta[volta unclosed]{c}{d}{2}
\end{tmstaff}
\end{tmline}
\end{codeexample}
\begin{command}{\tmvoltacoordinate\opt{\oarg{options}}\marg{coordinate 1}\marg{coordinate 2}\marg{number}}
  Draw a volta line between \meta{coordinate 1} and \meta{coordinate 2}.
\end{command}
\begin{codeexample}[]
\begin{tmline}
  \begin{tmstaff}{g}{}
    \tmbarlineinline{4}
    \tmquarter{5}{C4}{}\tmquarter{6}{D4}{}\tmquarter{7}{E4}{}
    \tmendrepeatbarlineinline{8}
    \tmquarter{9}{F4}{}\tmquarter{10}{G4}{}\tmquarter{11}{A4}{}
    \tmvoltacoordinate{4.1,1}{7.9,1}{1}\tmvoltacoordinate[volta unclosed]{8.1,1}{10.5,1}{2}
  \end{tmstaff}
\end{tmline}
\end{codeexample}
\subsection{Octave lines}\label{sec:line:octave}
\begin{command}{\tmoctave\opt{\oarg{options}}\marg{note 1}\marg{note 2}\marg{type}}
  Draw an octave line between \meta{note 1} and \meta{note 2}. \meta{type} can 
  be one of the following values: |8va|, |8vb|, |15ma|, |15mb|.
\end{command}
\begin{codeexample}[]
\begin{tmline}
\begin{tmstaff}{g}{}
  \tmquarter{4}{C4}{a} \tmquarter{5}{D4}{b} \tmoctave{a}{b}{8va}
  \tmquarter{6}{E4}{a} \tmquarter{7}{F4}{b} \tmoctave{a}{b}{8vb}
  \tmquarter{8}{G4}{a} \tmquarter{9}{A4}{b} \tmoctave{a}{b}{15ma}
  \tmquarter{10}{B4}{a}\tmquarter{11}{C5}{b}\tmoctave{a}{b}{15mb}
\end{tmstaff}
\end{tmline}
\end{codeexample}
\begin{command}{\tmoctavecoordinate\opt{\oarg{options}}\marg{coordinate 1}\marg{coordinate 2}\marg{type}}
  Draw an octave line between \meta{coordinate 1} and \meta{coordinate 2}.
\end{command}
\begin{codeexample}[]
\begin{tmline}
\begin{tmstaff}{g}{}
  \tmoctavecoordinate{1,1}{\linewidth-2mm,1}{8va}
\end{tmstaff}
\end{tmline}
\end{codeexample}
\subsection{Pedal lines}\label{sec:line:ped}
\begin{command}{\tmpedal\opt{\oarg{options}}\marg{note 1}\marg{note 2}}
  Draw a pedal line not ended with a star (\tikz\pic{tm-ped-star};) between 
  \meta{note 1} and \meta{note 2}.
\end{command}
\begin{command}{\tmpedalstar\opt{\oarg{options}}\marg{note 1}\marg{note 2}}
  Draw a pedal line ended with a star between \meta{note 1} and \meta{note 2}
\end{command}
\begin{codeexample}[]
\begin{tmline}
\begin{tmstaff}{g}{}
  \tmquarter{2}{C4}{a}\tmquarter{4}{D4}{b}\tmpedal{a}{b}
  \tmquarter{6}{E4}{a}\tmpedal{a}{a}
  \tmquarter{8}{C4}{a}\tmquarter{10}{D4}{b}\tmpedalstar{a}{b}
  \tmquarter{12}{E4}{a}\tmpedalstar{a}{a}
\end{tmstaff}
\end{tmline}
\end{codeexample}
\begin{command}{\tmpedalcoordinate\opt{\oarg{options}}\marg{coordinate 1}\marg{coordinate 2}}
  Draw a pedal line not ended with a star between \meta{coordinate 1} and 
  \meta{coordinate 2}.
\end{command}
\begin{command}{\tmpedalstarcoordinate\opt{\oarg{options}}\marg{coordinate 1}\marg{coordinate 2}}
  Draw a pedal line ended with a star between \meta{coordinate 1} and 
  \meta{coordinate 2}.
\end{command}
\begin{codeexample}[]
\begin{tmline}
\begin{tmstaff}{g}{}
  \tmpedalcoordinate{2,-1}{4,-1}\tmpedalcoordinate{6,-1}{6,-1}
  \tmpedalstarcoordinate{8,-1}{10,-1}\tmpedalstarcoordinate{12,-1}{12,-1}
\end{tmstaff}
\end{tmline}
\end{codeexample}
\subsection{Trill lines}\label{sec:line:trill}
\begin{command}{\tmtrillline\opt{\oarg{options}}\marg{note 1}\marg{note 2}}
  Draw a trill line from \meta{note 1} to \emph{the left} of \meta{note 2}.
\end{command}
\begin{codeexample}[]
\begin{tmline}
\begin{tmstaff}{g}{}
  \tmquarter{4}{C5}{a}\tmquarter{7}{D5}{b}
  \tmtrillline{a}{b}% Note that this doesn't get over note b.
\end{tmstaff}
\end{tmline}
\end{codeexample}
\begin{command}{\tmtrilllinecoordinate\opt{\oarg{options}}\marg{coordinate 1}\marg{coordinate 2}}
  Draw a trill line from \meta{coordinate 1} to \emph{the left} of \meta{coordinate 2}.
\end{command}
\begin{codeexample}[]
\begin{tmline}
\begin{tmstaff}{g}{}
  \tmquarter{4}{C5}{}\tmbarlineinline{7}\tmtrilllinecoordinate{4,.6}{7,.6}
\end{tmstaff}
\end{tmline}
\end{codeexample}
\subsection{Tuplets}\label{sec:line:tuplet}
\begin{command}{\tmtuplets\opt{\oarg{options}}\marg{note 1}\marg{note 2}\marg{number}}
  Draw a \meta{number}-th tuplet between \meta{note 1} and \meta{note 2}. 
  By default it will be drawn above the notes, but you can use |reversed| to draw 
  them below the notes.
\end{command}
\begin{codeexample}[]
\begin{tmline}
\begin{tmstaff}{g}{}
  \tmquarter{3}{C4}{a}\tmquarter{4}{E4}{b}
  \begin{tmbeam}
    \tmbeamnote{6}{D4}{1}{c}\tmbeamnote{6.5}{F4}{1}{}\tmbeamnote{7}{F4}{1}{d}
  \end{tmbeam}
  \tmtuplets[green]{a}{b}{2}\tmtuplets[reversed,red]{c}{d}{3}
\end{tmstaff}
\end{tmline}
\end{codeexample}
