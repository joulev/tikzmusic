\def\tm@note@min{52}
\def\tm@note@max{1}
\def\tm@string@center{center}

\pgfmathsetmacro\tm@draw@extra@lines@shift@right{2*\tm@notewidth}
\pgfmathsetmacro\tm@draw@extra@lines@shift@center{0}
\pgfmathsetmacro\tm@draw@extra@lines@shift@left{-2*\tm@notewidth}
\def\tm@draw@extra@lines#1#2#3#4{
  \pgfmathtruncatemacro\tm@extra@lines{div(#2-4,2)}
  \ifnum#2>5
    \foreach \tm@i [parse=true] in {3,...,{2+div(#2-4,2)}} 
      \draw[color/.expanded=\tm@linecolor,xshift=#4cm] (#1,\tm@i*.2) ++ (-#3,0) -- ++ (2*#3,0);
  \fi
  \ifnum#2<-5
    \foreach \tm@i [parse=true] in {3,...,{2+div(-#2-4,2)}}
      \draw[color/.expanded=\tm@linecolor,xshift=#4cm] (#1,-\tm@i*.2) ++ (-#3,0) -- ++ (2*#3,0);
  \fi
}


\pgfmathsetmacro\tm@note@getnumber@A{6}
\pgfmathsetmacro\tm@note@getnumber@B{7}
\pgfmathsetmacro\tm@note@getnumber@C{1}
\pgfmathsetmacro\tm@note@getnumber@D{2}
\pgfmathsetmacro\tm@note@getnumber@E{3}
\pgfmathsetmacro\tm@note@getnumber@F{4}
\pgfmathsetmacro\tm@note@getnumber@G{5}
\def\tm@note@getnumber<#1#2>{%
  \pgfmathtruncatemacro\tm@note@getnumber@fi{\csname tm@note@getnumber@#1\endcsname}%
  \pgfmathtruncatemacro\tm@note@getnumber@se{#2}%
  \pgfmathtruncatemacro\tm@note@getnumber@result{2+\tm@note@getnumber@fi+(\tm@note@getnumber@se-1)*7}%
}

%%%%%%%%%%%%%%%%%%%%%%%%%%%%%
% OBSOLETE
%\newcommand\tmappendaccidental[4][]{
%  \begingroup
%    \tmset{#1}
%    \iftm@usenotecolor
%      \def\tm@color{\@nameuse{tm@note@\detokenize{#2}@clr}}
%    \fi
%    \pic[color/.expanded=\tm@color,shift={(\tm@shift@start@x,\tm@shift@start@y)}] at ([xshift=-3mm]#2-#3) {tm-#4};
%  \endgroup
%}
%\newcommand\tmadddot[3][]{
%  \begingroup
%    \tmset{#1}
%    \iftm@usenotecolor
%      \def\tm@color{\@nameuse{tm@note@\detokenize{#2}@clr}}
%    \fi
%    \edef\tm@notelist{\@nameuse{tm@note@\detokenize{#2}@notes}}
%    \foreach \tm@i in \tm@notelist {
%      \expandafter\tm@note@getnumber\expandafter<\tm@i>
%      \pgfmathtruncatemacro\tm@note@diff{\tm@note@getnumber@result-\tm@note@mid@clef}
%      \pgfmathsetmacro\tm@note@dot@shift{ifthenelse(mod(\tm@note@diff,2)==0,1mm,0mm)}
%      \foreach \tm@j in {1,...,#3} {
%        \fill[color/.expanded=\tm@color] 
%          ([xshift=2.5mm,yshift=\tm@note@dot@shift]#2-\tm@i) ++ ({(\tm@j-1)*.1},0) circle (.4mm);
%      }
%    }
%  \endgroup
%}

\def\tm@note@currentnote{}
\def\tm@note@testargument{\@ifnextchar[\tm@note@testargument@yes\tm@note@testargument@no}
\def\tm@note@testargument@yes[#1]#2><#3>{
  \def\tm@note@currentnote{#2}
  \tm@note@getnumber<#2>
  \tmset{#1}
}
\def\tm@note@testargument@no#1><#2>{
  \def\tm@note@currentnote{#1}
  \tm@note@getnumber<#1>
  \tmset{#2}
}
\def\tm@string@none{none}
\def\tm@note@accidental{none}
\def\tm@note@dot@number{0}
\def\tm@string@zero{0}
\newcommand\tm@head[5][]{
  \def\tm@note@min{52}
  \def\tm@note@max{1}
  \foreach \tm@i in {#3} {
    \expandafter\tm@note@testargument\tm@i><#1>
    \pgfmathtruncatemacro\tm@note@diff{\tm@note@getnumber@result-\tm@note@mid@clef}
    \ifx\tm@note@relativepos\tm@string@center
      \pgfmathparse{min(\tm@note@min,\tm@note@getnumber@result)}
      \global\let\tm@note@min\pgfmathresult
      \pgfmathparse{max(\tm@note@max,\tm@note@getnumber@result)}
      \global\let\tm@note@max\pgfmathresult
    \fi
    \csname tm@#5@head\endcsname[\tm@note@relativepos]{#2}{\tm@note@diff}
    \coordinate (#4-\tm@note@currentnote) at (#2,.1*\tm@note@diff);
    \ifx\tm@string@none\tm@note@accidental\else
      \pic[color/.expanded=\tm@color] at ([xshift/.evaluated=-3mm+\tm@note@accidental@xshift,yshift=\tm@note@accidental@yshift]#4-\tm@note@currentnote) {tm-\tm@note@accidental};
    \fi
    \ifx\tm@string@zero\tm@note@dot@number\else
      \foreach \tm@j in {1,...,\tm@note@dot@number} {
        \pgfmathsetmacro\tm@note@dot@lshift{ifthenelse(mod(\tm@note@diff,2)==0,1mm,0mm)}
        \fill[color/.expanded=\tm@color] 
          ([xshift/.evaluated=2.5mm+\tm@note@dot@xshift,yshift/.evaluated=\tm@note@dot@yshift+\tm@note@dot@lshift]#4-\tm@note@currentnote) ++ ({(\tm@j-1)*.1},0) circle (.4mm);
      }
    \fi
  }
  \coordinate (#4-center) at (#2,0);
  \pgfmathtruncatemacro\tm@note@min@diff{\tm@note@min-\tm@note@mid@clef}
  \pgfmathtruncatemacro\tm@note@max@diff{\tm@note@max-\tm@note@mid@clef}
  \pgfmathsetmacro\tm@note@average@diff{(\tm@note@min@diff+\tm@note@max@diff)/2}
  \global\@nameedef{tm@note@\detokenize{#4}@max}{\tm@note@max@diff}
  \global\@nameedef{tm@note@\detokenize{#4}@min}{\tm@note@min@diff}
  \global\@nameedef{tm@note@\detokenize{#4}@avg}{\tm@note@average@diff}
  \global\@nameedef{tm@note@\detokenize{#4}@pos}{#2}
  \global\@nameedef{tm@note@\detokenize{#4}@clr}{\tm@color}
  \global\@nameedef{tm@note@\detokenize{#4}@notes}{#3}
  \def\tm@string@whole{whole}
  \def\tm@string@aux{#5}
  \ifx\tm@string@aux\tm@string@whole
    \coordinate (#4-tail) at (#2,.1*\tm@note@max@diff);
  \fi
}


%%%%%%%%%%%%%%%%%%%%%%%%%%%%%
% WHOLE NOTE
\newcommand\tm@whole@head[3][center]{%
  \tm@draw@extra@lines{#2}{#3}{.25}{\csname tm@draw@extra@lines@shift@#1\endcsname}
  \pic[color/.expanded=\tm@color] at (#2,#3*0.1) {tm-whole-note-#1};
}
\newcommand\tmwhole[4][]{
  \begingroup
    \tmset{#1}
    \tm@head[#1]{#2}{#3}{#4}{whole}
  \endgroup
  \def\tm@note@average@diff{0}
}

\def\tm@note@stemup#1#2{
  \draw[xshift=#2cm,color/.expanded=\tm@color] 
    (\tm@notewidth,.1*\tm@note@min@diff) -- 
    ([yshift=\tm@notelength]\tm@notewidth,.1*\tm@note@max@diff) coordinate (#1-tail);
  \iftm@note@gracestrike
    \draw[xshift=#2cm,color/.expanded=\tm@color] 
      ([yshift/.evaluated=\tm@notelength/2-1mm*\tm@scale]0,.1*\tm@note@max@diff) -- 
      ([yshift/.evaluated=\tm@notelength/2+1mm*\tm@scale]3.5*\tm@notewidth,.1*\tm@note@max@diff);
  \fi
  \global\@nameedef{tm@note@\detokenize{#1}@dir}{up}
}
\def\tm@note@stemdown#1#2{
  \draw[xshift=#2cm,color/.expanded=\tm@color] 
    (-\tm@notewidth,.1*\tm@note@max@diff) -- 
    ([yshift=-\tm@notelength]-\tm@notewidth,.1*\tm@note@min@diff) coordinate (#1-tail);
  \iftm@note@gracestrike
    \draw[xshift=#2cm,color/.expanded=\tm@color] 
      ([yshift/.evaluated=-\tm@notelength/2+1mm*\tm@scale]-2*\tm@notewidth,.1*\tm@note@max@diff) -- 
      ([yshift/.evaluated=-\tm@notelength/2-1mm*\tm@scale]1.5*\tm@notewidth,.1*\tm@note@max@diff);
  \fi
  \global\@nameedef{tm@note@\detokenize{#1}@dir}{down}
}

%%%%%%%%%%%%%%%%%%%%%%%%%%%%%%
% HALF NOTE
\newcommand\tm@half@head[3][center]{%
  \tm@draw@extra@lines{#2}{#3}{.2}{\csname tm@draw@extra@lines@shift@#1\endcsname}
  \pic[color/.expanded=\tm@color] at (#2,#3*0.1) {tm-half-note-head-#1};
}
\newcommand\tmhalf[4][]{
  \begingroup
    \tmset{#1}
    \tm@head[#1]{#2}{#3}{#4}{half}
    \ifdim\tm@note@average@diff pt<0pt
      \iftm@reversed\tm@note@stemdown{#4}{#2}\else\tm@note@stemup{#4}{#2}\fi
    \else
      \iftm@reversed\tm@note@stemup{#4}{#2}\else\tm@note@stemdown{#4}{#2}\fi
    \fi
  \endgroup
  \def\tm@note@average@diff{0}
}


%%%%%%%%%%%%%%%%%%%%%%%%%%%%%%%%%
% QUARTER NOTE
\newcommand\tm@quarter@head[3][center]{%
  \tm@draw@extra@lines{#2}{#3}{.2}{\csname tm@draw@extra@lines@shift@#1\endcsname}
  \pic[color/.expanded=\tm@color] at (#2,#3*0.1) {tm-quarter-note-head-#1};
}
\newcommand\tmquarter[4][]{
  \begingroup
    \tmset{#1}
    \tm@head[#1]{#2}{#3}{#4}{quarter}
    \ifdim\tm@note@average@diff pt<0pt
      \iftm@reversed\tm@note@stemdown{#4}{#2}\else\tm@note@stemup{#4}{#2}\fi
    \else
      \iftm@reversed\tm@note@stemup{#4}{#2}\else\tm@note@stemdown{#4}{#2}\fi
    \fi
  \endgroup
  \def\tm@note@average@diff{0}
}



%%%%%%%%%%%%%%%%%%%%%%%%%%%%%%%%%
% EIGHTH NOTE
\newcommand\tmeighth[4][]{
  \begingroup
    \tmset{#1}
    \tm@head[#1]{#2}{#3}{#4}{quarter}
    \ifdim\tm@note@average@diff pt<0pt
      \iftm@reversed
        \tm@note@stemdown{#4}{#2}
        \pic[color/.expanded=\tm@color] at (#4-tail) {tm-note-flag-down};
      \else
        \tm@note@stemup{#4}{#2}
        \pic[color/.expanded=\tm@color] at (#4-tail) {tm-note-flag-up};
      \fi
    \else
      \iftm@reversed
        \tm@note@stemup{#4}{#2}
        \pic[color/.expanded=\tm@color] at (#4-tail) {tm-note-flag-up};
      \else
        \tm@note@stemdown{#4}{#2}
        \pic[color/.expanded=\tm@color] at (#4-tail) {tm-note-flag-down};
      \fi
    \fi
  \endgroup
  \def\tm@note@average@diff{0}
}



%%%%%%%%%%%%%%%%%%%%%%%%%%%%%%%%%
% MORE THAN EIGHTH NOTE
\def\tm@insert@flag@up#1#2{
  \foreach \tm@i in {1,...,#2} {
    \pgfmathsetmacro\tm@insert@flag@shift@amount{-3mm+1.5mm*\tm@i}
    \pic[color/.expanded=\tm@color] 
      at ([yshift=\tm@insert@flag@shift@amount*\tm@scale]#1-tail) {tm-note-flag-up};
  }
}
\def\tm@insert@flag@down#1#2{
  \foreach \tm@i in {1,...,#2} {
    \pgfmathsetmacro\tm@insert@flag@shift@amount{3mm-1.5mm*\tm@i}
    \pic[color/.expanded=\tm@color] 
      at ([yshift=\tm@insert@flag@shift@amount*\tm@scale]#1-tail) {tm-note-flag-down};
  }
}

\newcommand\tmmorethaneighth[5][]{
  \begingroup
    \tmset{#1}
    \tm@head[#1]{#2}{#3}{#5}{quarter}
    \ifdim\tm@note@average@diff pt<0pt
      \iftm@reversed
        \tm@note@stemdown{#5}{#2}
        \tm@insert@flag@down{#5}{#4}
      \else
        \tm@note@stemup{#5}{#2}
        \tm@insert@flag@up{#5}{#4}
      \fi
    \else
      \iftm@reversed
        \tm@note@stemup{#5}{#2}
        \tm@insert@flag@up{#5}{#4}
      \else
        \tm@note@stemdown{#5}{#2}
        \tm@insert@flag@down{#5}{#4}
      \fi
    \fi
  \endgroup
  \def\tm@note@average@diff{0}
}





%%%%%%%%%%%%%%%%%%%%%%%%%%%%%%%%%%%%
% BEAM
% <By marmot>
\def\tm@beam@arr@pushback#1{\ifcsname tm@beam@array\endcsname\edef\tm@beam@array{\tm@beam@array,#1}
\else\edef\tm@beam@array{#1}\fi}
% </By marmot>
% Beam heading up
\newif\iftm@beam@maxatmiddle
\newif\iftm@beam@minatmiddle
\newif\iftm@beam@firstpassed
\newif\iftm@beam@firstnoteunflagged
\newif\iftm@beam@up
\newenvironment{tmbeam}[1][]{
  \begin{scope}
    \tmset{#1}
    \tm@beam@uptrue
    \tm@beam@maxatmiddlefalse
    \tm@beam@minatmiddlefalse
    \tm@beam@firstpassedfalse
    \tm@beam@firstnoteunflaggedtrue
    \def\tm@beam@first@index{1}
    \def\tm@beam@last@index{1}
    \def\tm@beam@secondlast@index{1}
    \pgfmathsetmacro\tm@beam@first{0}
    \pgfmathsetmacro\tm@beam@min{52}
    \def\tm@beam@min@index{1}
    \pgfmathsetmacro\tm@beam@secondmin{52}
    \def\tm@beam@secondmin@index{1}
    \pgfmathsetmacro\tm@beam@max{1}
    \def\tm@beam@max@index{1}
    \pgfmathsetmacro\tm@beam@secondmax{1}
    \def\tm@beam@secondmax@index{1}
}{
    \tm@beam@drawbeamup
  \end{scope}
}
% Beam heading down
\newenvironment{tmbeam*}[1][]{
  \begin{scope}
    \tmset{#1}
    \tm@beam@upfalse
    \tm@beam@maxatmiddlefalse
    \tm@beam@minatmiddlefalse
    \tm@beam@firstpassedfalse
    \tm@beam@firstnoteunflaggedtrue
    \def\tm@beam@first@index{1}
    \def\tm@beam@last@index{1}
    \def\tm@beam@secondlast@index{1}
    \pgfmathsetmacro\tm@beam@first{0}
    \pgfmathsetmacro\tm@beam@min{52}
    \def\tm@beam@min@index{1}
    \pgfmathsetmacro\tm@beam@secondmin{52}
    \def\tm@beam@secondmin@index{1}
    \pgfmathsetmacro\tm@beam@max{1}
    \def\tm@beam@max@index{1}
    \pgfmathsetmacro\tm@beam@secondmax{1}
    \def\tm@beam@secondmax@index{1}
}{
    \tm@beam@drawbeamdown
  \end{scope}
}
%\tmbeamnote[]{xpos}{notecode}{no of flags}{name}
\newcommand\tmbeamnote[5][]{
  \iftm@beam@up
    \tm@beam@head@max[#1]{#2}{#3}{#5}
    \global\@nameedef{tm@note@\detokenize{#5}@dir}{up}
  \else
    \tm@beam@head@min[#1]{#2}{#3}{#5}
    \global\@nameedef{tm@note@\detokenize{#5}@dir}{down}
  \fi
  \tm@beam@arr@pushback{#5}
  \expandafter\def\csname tm@beam@numberofflags@#5\endcsname{#4}
}
\newcommand\tm@beam@head[4][]{
  \def\tm@note@min{52}
  \def\tm@note@max{1}
  \begingroup
    \tmset{#1}
    \foreach \tm@i in {#3} {
      \expandafter\tm@note@testargument\tm@i><#1>
      \pgfmathtruncatemacro\tm@note@diff{\tm@note@getnumber@result-\tm@note@mid@clef}
      \ifx\tm@note@relativepos\tm@string@center
        \pgfmathparse{min(\tm@note@min,\tm@note@getnumber@result)}
        \global\let\tm@note@min\pgfmathresult
        \pgfmathparse{max(\tm@note@max,\tm@note@getnumber@result)}
        \global\let\tm@note@max\pgfmathresult
      \fi
      \tm@quarter@head[\tm@note@relativepos]{#2}{\tm@note@diff}
      \coordinate (#4-\tm@note@currentnote) at (#2,.1*\tm@note@diff);
      \ifx\tm@string@none\tm@note@accidental\else
        \pic[color/.expanded=\tm@color] at ([xshift/.evaluated=-3mm+\tm@note@accidental@xshift,yshift=\tm@note@accidental@yshift]#4-\tm@note@currentnote) {tm-\tm@note@accidental};
      \fi
      \ifx\tm@string@zero\tm@note@dot@number\else
        \foreach \tm@j in {1,...,\tm@note@dot@number} {
          \pgfmathsetmacro\tm@note@dot@lshift{ifthenelse(mod(\tm@note@diff,2)==0,1mm,0mm)}
          \fill[color/.expanded=\tm@color] 
            ([xshift/.evaluated=2.5mm+\tm@note@dot@xshift,yshift/.evaluated=\tm@note@dot@yshift+\tm@note@dot@lshift]#4-\tm@note@currentnote) ++ ({(\tm@j-1)*.1},0) circle (.4mm);
        }
      \fi
    }
    \coordinate (#4-center) at (#2,0);
    \global\@nameedef{tm@note@\detokenize{#4}@clr}{\tm@color}
  \endgroup
  \pgfmathtruncatemacro\tm@note@min@diff{\tm@note@min-\tm@note@mid@clef}
  \pgfmathtruncatemacro\tm@note@max@diff{\tm@note@max-\tm@note@mid@clef}
  \pgfmathsetmacro\tm@note@average@diff{(\tm@note@min@diff+\tm@note@max@diff)/2}
  \global\@nameedef{tm@note@\detokenize{#4}@max}{\tm@note@max@diff}
  \global\@nameedef{tm@note@\detokenize{#4}@min}{\tm@note@min@diff}
  \global\@nameedef{tm@note@\detokenize{#4}@avg}{\tm@note@average@diff}
  \global\@nameedef{tm@note@\detokenize{#4}@pos}{#2}
  \global\@nameedef{tm@note@\detokenize{#4}@notes}{#3}
}
\newcommand\tm@beam@head@max[4][]{
  \tm@beam@head[#1]{#2}{#3}{#4}
  \tm@beam@note@getmax{\tm@note@max}{#4}
  \coordinate (#4-@aux1) at (#2+\tm@notewidth,\tm@note@min@diff*.1);
  \coordinate (#4-@aux2) at ([yshift=\tm@notelength]#2+\tm@notewidth,\tm@note@max@diff*.1);
}
\newcommand\tm@beam@head@min[4][]{
  \tm@beam@head[#1]{#2}{#3}{#4}
  \tm@beam@note@getmin{\tm@note@min}{#4}
  \coordinate (#4-@aux1) at (#2-\tm@notewidth,\tm@note@max@diff*.1);
  \coordinate (#4-@aux2) at ([yshift=-\tm@notelength]#2-\tm@notewidth,\tm@note@min@diff*.1);
}



\def\tm@beam@note@getmax#1#2{
  \pgfmathparse{max(#1,\tm@beam@max)}
  \pgfmathsetmacro\tm@beam@aux@value{\tm@beam@max}
  \ifdim\pgfmathresult pt=#1pt
    \global\let\tm@beam@max\pgfmathresult
    \edef\tm@beam@secondmax@index{\tm@beam@max@index}
    \def\tm@beam@max@index{#2}
    \pgfmathsetmacro\tm@beam@secondmax{\tm@beam@aux@value}
    \tm@beam@maxatmiddlefalse
    \iftm@beam@firstpassed\else
      \pgfmathsetmacro\tm@beam@first{\tm@beam@max}
      \tm@beam@firstpassedtrue
      \def\tm@beam@first@index{#2}
    \fi
  \else
    \pgfmathparse{max(#1,\tm@beam@secondmax)}
    \ifdim\pgfmathresult pt=#1pt
      \global\let\tm@beam@secondmax\pgfmathresult
      \def\tm@beam@secondmax@index{#2}
    \fi
    \tm@beam@maxatmiddletrue
  \fi
}
\def\tm@beam@drawbeamup{
  \ifdim\tm@beam@first pt=\tm@beam@max pt
    \tm@beam@maxatmiddlefalse
  \fi
  \iftm@beam@maxatmiddle
    \coordinate (tm@a) at (\tm@beam@max@index-@aux2);
    \coordinate (tm@b) at ([xshift=1mm]tm@a);
  \else
    \pgfmathsetmacro\tm@beam@slope@min{100}%just take a sufficiently large number
    \foreach \tm@i in \tm@beam@array {
      \coordinate[overlay] (x) at ($(\tm@beam@max@index-@aux2)-(\tm@i-@aux2)$);
      \path[overlay] (x);
      \pgfgetlastxy{\tm@x}{\tm@y}
      \ifdim\tm@x=0pt
      \else
        \pgfmathparse{(abs(\tm@y))/(abs(\tm@x))}
        \ifdim\tm@beam@slope@min pt>\pgfmathresult pt
          \global\let\tm@beam@slope@min\pgfmathresult
          \coordinate (tm@a) at (\tm@beam@max@index-@aux2);
          \coordinate (tm@b) at (\tm@i-@aux2);
        \fi
        \ifdim\tm@beam@slope@min pt>0.2pt
          \ifdim\tm@x>0pt
            \coordinate (tm@b) at ([shift={(11.30993247:.1)}]tm@a);
          \else
            \coordinate (tm@b) at ([shift={(-11.30993247:.1)}]tm@a);
          \fi
        \fi
      \fi
    }
  \fi
  \foreach \tm@i in \tm@beam@array {
    \path (intersection cs:first line={(\tm@i-@aux1) -- (\tm@i-@aux2)},second line={(tm@a)--(tm@b)})
      coordinate (\tm@i-beamintersection);
    \draw[color/.expanded=\tm@color] (\tm@i-@aux1) -- (\tm@i-beamintersection) 
      coordinate (\tm@i-tail);
  }
  \foreach \tm@i [remember=\tm@i as \tm@j (initially \tm@beam@first@index)] in \tm@beam@array {
    \def\tm@beam@j@flags{\expandafter\csname tm@beam@numberofflags@\tm@j\endcsname}
    \def\tm@beam@i@flags{\expandafter\csname tm@beam@numberofflags@\tm@i\endcsname}
    \tm@beam@draw@betweennotes@up{\tm@j}{\tm@beam@j@flags}{\tm@i}{\tm@beam@i@flags}
    \global\edef\tm@beam@last@index{\tm@i}
    \global\edef\tm@beam@secondlast@index{\tm@j}
  }
  \def\tm@beam@x@flags{\expandafter\csname tm@beam@numberofflags@\tm@beam@last@index\endcsname}
  \tm@beam@draw@betweennotes@up@final{\tm@beam@secondlast@index}{\tm@beam@last@index}{\expandafter\csname tm@beam@numberofflags@\tm@beam@last@index\endcsname}
}
%\...{name of note 1}{number of note 1}{name of note 2}{number of note 2}
\def\tm@beam@draw@betweennotes@up#1#2#3#4{
  \ifnum#2>#4
    \foreach \tm@x [count=\tm@y from 0] in {1,...,#4} {
      \coordinate (@beamintersection@aux1) at ([yshift=-\tm@y cm*0.15]#1-beamintersection);
      \coordinate (@beamintersection@aux2) at ([yshift=-\tm@y cm*0.15]#3-beamintersection);
      \fill[color/.expanded=\tm@color] (@beamintersection@aux1) -- ++ (0,-.08) -- 
        ([yshift=-.8mm]@beamintersection@aux2) -- ++ (0,.08) -- cycle;
    }
    \iftm@beam@firstnoteunflagged
      \coordinate (@beamintersection@aux@mid) at ($(#1-beamintersection)!.3!(#3-beamintersection)$);
      \foreach \tm@x [count=\tm@y from 0] in {1,...,#2} {
        \coordinate (@beamintersection@aux1) at ([yshift=-\tm@y cm*0.15]#1-beamintersection);
        \coordinate (@beamintersection@aux2) at ([yshift=-\tm@y cm*0.15]@beamintersection@aux@mid);
        \fill[color/.expanded=\tm@color] (@beamintersection@aux1) -- ++ (0,-.08) -- 
          ([yshift=-.8mm]@beamintersection@aux2) -- ++ (0,.08) -- cycle;
      }
    \fi
    \tm@beam@firstnoteunflaggedfalse
  \fi
  \ifnum#2=#4
    \foreach \tm@x [count=\tm@y from 0] in {1,...,#4} {
      \coordinate (@beamintersection@aux1) at ([yshift=-\tm@y cm*0.15]#1-beamintersection);
      \coordinate (@beamintersection@aux2) at ([yshift=-\tm@y cm*0.15]#3-beamintersection);
      \fill[color/.expanded=\tm@color] (@beamintersection@aux1) -- ++ (0,-.08) -- 
        ([yshift=-.8mm]@beamintersection@aux2) -- ++ (0,.08) -- cycle;
    }
    \tm@beam@firstnoteunflaggedfalse
  \fi
  \ifnum#2<#4
    \foreach \tm@x [count=\tm@y from 0] in {1,...,#2} {
      \coordinate (@beamintersection@aux1) at ([yshift=-\tm@y cm*0.15]#1-beamintersection);
      \coordinate (@beamintersection@aux2) at ([yshift=-\tm@y cm*0.15]#3-beamintersection);
      \fill[color/.expanded=\tm@color] (@beamintersection@aux1) -- ++ (0,-.08) -- 
        ([yshift=-.8mm]@beamintersection@aux2) -- ++ (0,.08) -- cycle;
    }
    \tm@beam@firstnoteunflaggedtrue
  \fi
}
% \...{note 1}{note 2}{flag 2}
\def\tm@beam@draw@betweennotes@up@final#1#2#3{
  \iftm@beam@firstnoteunflagged
    \coordinate (@beamintersection@aux@mid) at ($(#2-beamintersection)!.3!(#1-beamintersection)$);
    \foreach \tm@x [count=\tm@y from 0] in {1,...,#3} {
      \coordinate (@beamintersection@aux1) at ([yshift=-\tm@y cm*0.15]@beamintersection@aux@mid);
      \coordinate (@beamintersection@aux2) at ([yshift=-\tm@y cm*0.15]#2-beamintersection);
      \fill[color/.expanded=\tm@color] (@beamintersection@aux1) -- ++ (0,-.08) -- 
        ([yshift=-.8mm]@beamintersection@aux2) -- ++ (0,.08) -- cycle;
    }
  \fi
}






\def\tm@beam@note@getmin#1#2{
  \pgfmathparse{min(#1,\tm@beam@min)}
  \pgfmathsetmacro\tm@beam@aux@value{\tm@beam@min}
  \ifdim\pgfmathresult pt=#1pt
    \global\let\tm@beam@min\pgfmathresult
    \edef\tm@beam@secondmin@index{\tm@beam@min@index}
    \def\tm@beam@min@index{#2}
    \pgfmathsetmacro\tm@beam@secondmin{\tm@beam@aux@value}
    \tm@beam@minatmiddlefalse
    \iftm@beam@firstpassed\else
      \pgfmathsetmacro\tm@beam@first{\tm@beam@min}
      \tm@beam@firstpassedtrue
      \def\tm@beam@first@index{#2}
    \fi
  \else
    \pgfmathparse{min(#1,\tm@beam@secondmin)}
    \ifdim\pgfmathresult pt=#1pt
      \global\let\tm@beam@secondmin\pgfmathresult
      \def\tm@beam@secondmin@index{#2}
    \fi
    \tm@beam@minatmiddletrue
  \fi
}
\def\tm@beam@drawbeamdown{
  \ifdim\tm@beam@first pt=\tm@beam@min pt
    \tm@beam@minatmiddlefalse
  \fi
  \iftm@beam@minatmiddle
    \coordinate (tm@a) at (\tm@beam@min@index-@aux2);
    \coordinate (tm@b) at ([xshift=1mm]tm@a);
  \else
    \pgfmathsetmacro\tm@beam@slope@min{100}%just take a sufficiently large number
    \foreach \tm@i in \tm@beam@array {
      \coordinate[overlay] (x) at ($(\tm@beam@min@index-@aux2)-(\tm@i-@aux2)$);
      \path[overlay] (x);
      \pgfgetlastxy{\tm@x}{\tm@y}
      \ifdim\tm@x=0pt
      \else
        \pgfmathparse{(abs(\tm@y))/(abs(\tm@x))}
        \ifdim\tm@beam@slope@min pt>\pgfmathresult pt
          \global\let\tm@beam@slope@min\pgfmathresult
          \coordinate (tm@a) at (\tm@beam@min@index-@aux2);
          \coordinate (tm@b) at (\tm@i-@aux2);
        \fi
        \ifdim\tm@beam@slope@min pt>0.2pt
          \ifdim\tm@x>0pt
            \coordinate (tm@b) at ([shift={(-11.30993247:.1)}]tm@a);
          \else
            \coordinate (tm@b) at ([shift={(11.30993247:.1)}]tm@a);
          \fi
        \fi
      \fi
    }
  \fi
  \foreach \tm@i in \tm@beam@array {
    \path (intersection cs:first line={(\tm@i-@aux1) -- (\tm@i-@aux2)},second line={(tm@a)--(tm@b)})
      coordinate (\tm@i-beamintersection);
    \draw[color/.expanded=\tm@color] (\tm@i-@aux1) -- (\tm@i-beamintersection);
  }
  \foreach \tm@i [remember=\tm@i as \tm@j (initially \tm@beam@first@index)] in \tm@beam@array {
    \def\tm@beam@j@flags{\expandafter\csname tm@beam@numberofflags@\tm@j\endcsname}
    \def\tm@beam@i@flags{\expandafter\csname tm@beam@numberofflags@\tm@i\endcsname}
    \tm@beam@draw@betweennotes@down{\tm@j}{\tm@beam@j@flags}{\tm@i}{\tm@beam@i@flags}
    \global\edef\tm@beam@last@index{\tm@i}
    \global\edef\tm@beam@secondlast@index{\tm@j}
  }
  \def\tm@beam@x@flags{\expandafter\csname tm@beam@numberofflags@\tm@beam@last@index\endcsname}
  \tm@beam@draw@betweennotes@down@final{\tm@beam@secondlast@index}{\tm@beam@last@index}{\expandafter\csname tm@beam@numberofflags@\tm@beam@last@index\endcsname}
}
%\...{name of note 1}{number of note 1}{name of note 2}{number of note 2}
\def\tm@beam@draw@betweennotes@down#1#2#3#4{
  \ifnum#2>#4
    \foreach \tm@x [count=\tm@y from 0] in {1,...,#4} {
      \coordinate (@beamintersection@aux1) at ([yshift=\tm@y cm*0.15]#1-beamintersection);
      \coordinate (@beamintersection@aux2) at ([yshift=\tm@y cm*0.15]#3-beamintersection);
      \fill[color/.expanded=\tm@color] (@beamintersection@aux1) -- ++ (0,.08) -- 
        ([yshift=.8mm]@beamintersection@aux2) -- ++ (0,-.08) -- cycle;
    }
    \iftm@beam@firstnoteunflagged
      \coordinate (@beamintersection@aux@mid) at ($(#1-beamintersection)!.3!(#3-beamintersection)$);
      \foreach \tm@x [count=\tm@y from 0] in {1,...,#2} {
        \coordinate (@beamintersection@aux1) at ([yshift=\tm@y cm*0.15]#1-beamintersection);
        \coordinate (@beamintersection@aux2) at ([yshift=\tm@y cm*0.15]@beamintersection@aux@mid);
        \fill[color/.expanded=\tm@color] (@beamintersection@aux1) -- ++ (0,.08) -- 
          ([yshift=.8mm]@beamintersection@aux2) -- ++ (0,-.08) -- cycle;
      }
    \fi
    \tm@beam@firstnoteunflaggedfalse
  \fi
  \ifnum#2=#4
    \foreach \tm@x [count=\tm@y from 0] in {1,...,#4} {
      \coordinate (@beamintersection@aux1) at ([yshift=\tm@y cm*0.15]#1-beamintersection);
      \coordinate (@beamintersection@aux2) at ([yshift=\tm@y cm*0.15]#3-beamintersection);
      \fill[color/.expanded=\tm@color] (@beamintersection@aux1) -- ++ (0,.08) -- 
        ([yshift=.8mm]@beamintersection@aux2) -- ++ (0,-.08) -- cycle;
    }
    \tm@beam@firstnoteunflaggedfalse
  \fi
  \ifnum#2<#4
    \foreach \tm@x [count=\tm@y from 0] in {1,...,#2} {
      \coordinate (@beamintersection@aux1) at ([yshift=\tm@y cm*0.15]#1-beamintersection);
      \coordinate (@beamintersection@aux2) at ([yshift=\tm@y cm*0.15]#3-beamintersection);
      \fill[color/.expanded=\tm@color] (@beamintersection@aux1) -- ++ (0,.08) -- 
        ([yshift=.8mm]@beamintersection@aux2) -- ++ (0,-.08) -- cycle;
    }
    \tm@beam@firstnoteunflaggedtrue
  \fi
}
% \...{note 1}{note 2}{flag 2}
\def\tm@beam@draw@betweennotes@down@final#1#2#3{
  \iftm@beam@firstnoteunflagged
    \coordinate (@beamintersection@aux@mid) at ($(#2-beamintersection)!.3!(#1-beamintersection)$);
    \foreach \tm@x [count=\tm@y from 0] in {1,...,#3} {
      \coordinate (@beamintersection@aux1) at ([yshift=\tm@y cm*0.15]@beamintersection@aux@mid);
      \coordinate (@beamintersection@aux2) at ([yshift=\tm@y cm*0.15]#2-beamintersection);
      \fill[color/.expanded=\tm@color] (@beamintersection@aux1) -- ++ (0,.08) -- 
        ([yshift=.8mm]@beamintersection@aux2) -- ++ (0,-.08) -- cycle;
    }
  \fi
}






%%%%%%%%%%%%%%%%%%%%%%%%%
% RESTS
\newcommand\tmwholerest[2][]{
  \begingroup
    \tmset{#1}
    \fill[color/.expanded=\tm@color,shift={($(0,\tm@lineshift*.1)+(\tm@shift@start@x,\tm@shift@start@y)$)}] (#2-.125,.08) rectangle ++ (.25,.12);
    \draw[color/.expanded=\tm@linecolor,shift={($(0,\tm@lineshift*.1)+(\tm@shift@start@x,\tm@shift@start@y)$)}] (#2-.2,.2) -- ++ (.4,0);
  \endgroup
}
\newcommand\tmhalfrest[2][]{
  \begingroup
    \tmset{#1}
    \fill[color/.expanded=\tm@color,shift={($(0,\tm@lineshift*.1)+(\tm@shift@start@x,\tm@shift@start@y)$)}] (#2-.125,0) rectangle ++ (.25,.12);
    \draw[color/.expanded=\tm@linecolor,shift={($(0,\tm@lineshift*.1)+(\tm@shift@start@x,\tm@shift@start@y)$)}] (#2-.2,0) -- ++ (.4,0);
  \endgroup
}
\newcommand\tmquarterrest[2][]{
  \begingroup
    \tmset{#1}
    \pic[color/.expanded=\tm@color,shift={(\tm@shift@start@x,\tm@shift@start@y)}] at (#2,\tm@lineshift*.1) {tm-quarter-note-rest};
  \endgroup
}
\newcommand\tmbelowquarterrest[3][]{
  \begingroup
    \tmset{#1}
    \pic[color/.expanded=\tm@color,shift={(\tm@shift@start@x,\tm@shift@start@y)}] at (#2,\tm@lineshift*.1) {tm-#3-note-rest};
  \endgroup
}
\newcommand\tmeighthrest[2][]{\tmbelowquarterrest[#1]{#2}{1}}
\newcommand\tmsixteenthrest[2][]{\tmbelowquarterrest[#1]{#2}{2}}
\newcommand\tmthirtysecondrest[2][]{\tmbelowquarterrest[#1]{#2}{3}}
\newcommand\tmsixtyfourthrest[2][]{\tmbelowquarterrest[#1]{#2}{4}}



%%%%%%%%%%%%%%%%%%%%%%%%%%
% GET NOTES

\def\tm@note@getnote#1{
  \@ifundefined{tm@note@\detokenize{#1}@avg}{
    \tm@error{No note named `#1' is known}
  }{
    \pgfmathsetmacro\tm@note@average@diff{\@nameuse{tm@note@\detokenize{#1}@avg}}
    \pgfmathtruncatemacro\tm@note@min@diff{\@nameuse{tm@note@\detokenize{#1}@min}}
    \pgfmathtruncatemacro\tm@note@max@diff{\@nameuse{tm@note@\detokenize{#1}@max}}
    \pgfmathsetmacro\tm@note@pos{\@nameuse{tm@note@\detokenize{#1}@pos}}
    \edef\tm@note@dir{\@nameuse{tm@note@\detokenize{#1}@dir}}
  }
}