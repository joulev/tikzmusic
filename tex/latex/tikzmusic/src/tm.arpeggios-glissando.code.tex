\newcommand\tmarpeggio[2][]{
  \begingroup
    \tmset{#1}
    \iftm@usenotecolor
      \def\tm@color{\@nameuse{tm@note@\detokenize{#2}@clr}}
    \fi
    \tm@note@getnote{#2}
    \draw[xshift=-4mm,decorate,decoration=tm@trillline,color/.expanded=\tm@color] 
      ([yshift/.evaluated=-1mm+\tm@shift@start@y,xshift=\tm@shift@start@x]\tm@note@pos,\tm@note@min@diff*.1) -- 
      ([yshift/.evaluated=2mm+\tm@shift@end@y,xshift=\tm@shift@end@x]\tm@note@pos,\tm@note@max@diff*.1);
  \endgroup
}
% TODO: Use /pgf/decoration/pre and /pgf/decoration/post instead of using a new decoration.
\newcommand\tmarpeggioup[2][]{
  \begingroup
    \tmset{#1}
    \iftm@usenotecolor
      \def\tm@color{\@nameuse{tm@note@\detokenize{#2}@clr}}
    \fi
    \tm@note@getnote{#2}
    \draw[xshift=-4mm,decorate,decoration=tm@trilllineto,color/.expanded=\tm@color] 
      ([yshift/.evaluated=-1mm+\tm@shift@start@y,xshift=\tm@shift@start@x]\tm@note@pos,\tm@note@min@diff*.1) -- 
      ([yshift/.evaluated=2mm+\tm@shift@end@y,xshift=\tm@shift@end@x]\tm@note@pos,\tm@note@max@diff*.1);
  \endgroup
}
\newcommand\tmarpeggiodown[2][]{
  \begingroup
    \tmset{#1}
    \iftm@usenotecolor
      \def\tm@color{\@nameuse{tm@note@\detokenize{#2}@clr}}
    \fi
    \tm@note@getnote{#2}
    \draw[xshift=-4mm,decorate,decoration=tm@trilllineto,color/.expanded=\tm@color] 
      ([yshift/.evaluated=2mm+\tm@shift@end@y,xshift=\tm@shift@end@x]\tm@note@pos,\tm@note@max@diff*.1) -- 
      ([yshift/.evaluated=-1mm+\tm@shift@start@y,xshift=\tm@shift@start@x]\tm@note@pos,\tm@note@min@diff*.1);
  \endgroup
}

\newcommand\tmglissandocoordinate[3][]{
  \begingroup
    \tmset{#1}
    \draw[decorate,decoration=tm@trillline,color/.expanded=\tm@color] 
      ([shift={(\tm@shift@start@x,\tm@shift@start@y)}]#2) -- ([shift={(\tm@shift@end@x,\tm@shift@end@y)}]#3) 
      \iftm@hidetext;\else node[midway,above,sloped,font=\itshape] {gliss.};\fi
  \endgroup
}
\newcommand\tmglissando[3][]{
  \begingroup
    \tmset{#1}
    \tm@note@getnote{#2}
    \coordinate (tm@aux1) at (\tm@note@pos,\tm@note@min@diff*.1);
    \tm@note@getnote{#3}
    \coordinate (tm@aux2) at (\tm@note@pos,\tm@note@max@diff*.1);
    \draw[decorate,decoration=tm@trillline,color/.expanded=\tm@color] 
      ([shift={(\tm@shift@start@x,\tm@shift@start@y)}]intersection cs:first line={(tm@aux1)--(tm@aux2)},second line={([xshift=3mm]tm@aux1)--([xshift=3mm,yshift=.5mm]tm@aux1)}) -- 
      ([shift={(\tm@shift@end@x,\tm@shift@end@y)}]intersection cs:first line={(tm@aux1)--(tm@aux2)},second line={([xshift=-3mm]tm@aux2)--([xshift=-3mm,yshift=.5mm]tm@aux2)})
      % This is a pretty strange syntax. I was surprised that it worked.
      \iftm@hidetext; \else node[midway,above,sloped,font=\itshape] {gliss.}; \fi
  \endgroup
}