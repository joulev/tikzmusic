%%%%%%%%%%%%%%%%%%%%%%%%%%%%%%%%%%%
% SLUR

\newcommand\tmslur[3][]{
  \begingroup
    \tmset{#1}
    \iftm@reversed
      \tm@note@getnote{#2}\coordinate (tm@line@aux1) at (\tm@note@pos,\tm@note@max@diff*.1+.2);
      \tm@note@getnote{#3}\coordinate (tm@line@aux2) at (\tm@note@pos,\tm@note@max@diff*.1+.2);
      \draw[decorate,decoration={calligraphic curved parenthesis,amplitude=\tm@amplitude},very thick,pen colour/.expanded=\tm@color]
        ([shift={(\tm@shift@start@x,\tm@shift@start@y)}]tm@line@aux1) -- 
        ([shift={(\tm@shift@end@x,\tm@shift@end@y)}]tm@line@aux2);
    \else
      \tm@note@getnote{#2}\coordinate (tm@line@aux1) at (\tm@note@pos,\tm@note@min@diff*.1-.2);
      \tm@note@getnote{#3}\coordinate (tm@line@aux2) at (\tm@note@pos,\tm@note@min@diff*.1-.2);
      \draw[decorate,decoration={calligraphic curved parenthesis,amplitude=\tm@amplitude,mirror},very thick,pen colour/.expanded=\tm@color]
        ([shift={(\tm@shift@start@x,\tm@shift@start@y)}]tm@line@aux1) -- 
        ([shift={(\tm@shift@end@x,\tm@shift@end@y)}]tm@line@aux2);
    \fi
  \endgroup
}
% Custom coordinates
\newcommand\tmslurcoordinate[3][]{
  \begingroup
    \tmset{#1}
    \coordinate (tm@line@aux1) at ([shift={(\tm@shift@start@x,\tm@shift@start@y)}]#2);
    \coordinate (tm@line@aux2) at ([shift={(\tm@shift@end@x,\tm@shift@end@y)}]#3);
    \iftm@reversed
      \draw[decorate,decoration={calligraphic curved parenthesis,amplitude=\tm@amplitude},very thick,pen colour/.expanded=\tm@color]
        (tm@line@aux1) -- (tm@line@aux2);
    \else
      \draw[decorate,decoration={calligraphic curved parenthesis,amplitude=\tm@amplitude,mirror},very thick,pen colour/.expanded=\tm@color]
        (tm@line@aux1) -- (tm@line@aux2);
    \fi
  \endgroup
}

%%%%%%%%%%%%%%%%%%%%%%%%%%%%
% TIE - just a special slur

% Note: this is intended for use with multinote notes. If a note set has only a 
% single note, you should use slur.
\newcommand\tmtie[3][]{
  \begingroup
    \tmset{amplitude=1.5mm}% This is the default for tying
    \tmset{#1}
    \edef\tm@notelist{\@nameuse{tm@note@\detokenize{#2}@notes}}
    \edef\tm@note@avg{\@nameuse{tm@note@\detokenize{#2}@avg}}
    \edef\tm@note@pos@fi{\@nameuse{tm@note@\detokenize{#2}@pos}}
    \edef\tm@note@pos@se{\@nameuse{tm@note@\detokenize{#3}@pos}}
    \foreach \tm@i in \tm@notelist {
      \expandafter\tm@note@getnumber\expandafter<\tm@i>
      \pgfmathtruncatemacro\tm@note@diff{\tm@note@getnumber@result-\tm@note@mid@clef}
      \ifdim\tm@note@diff pt>\tm@note@avg pt
        % If it is larger than average: slur joining above
        \ifdim\tm@note@pos@fi pt<\tm@note@pos@se pt
          % First is the left, second is the right
          \coordinate (tm@line@aux1) at ([shift={(\tm@shift@start@x,\tm@shift@start@y)}]#2-\tm@i);
          \coordinate (tm@line@aux2) at ([shift={(\tm@shift@end@x,\tm@shift@end@y)}]#3-\tm@i);
        \else
          % First is the right
          \coordinate (tm@line@aux1) at ([shift={(\tm@shift@start@x,\tm@shift@start@y)}]#3-\tm@i);
          \coordinate (tm@line@aux2) at ([shift={(\tm@shift@end@x,\tm@shift@end@y)}]#2-\tm@i);
        \fi
        \draw[decorate,decoration={calligraphic curved parenthesis,amplitude=\tm@amplitude},very thick,pen colour/.expanded=\tm@color]
          ([shift={(2mm,1mm)}]tm@line@aux1) -- ([shift={(-2mm,1mm)}]tm@line@aux2);
      \else
        % If it is smaller than or equal to average: slur joining below
        \ifdim\tm@note@pos@fi pt<\tm@note@pos@se pt
          \coordinate (tm@line@aux1) at ([shift={(\tm@shift@start@x,\tm@shift@start@y)}]#2-\tm@i);
          \coordinate (tm@line@aux2) at ([shift={(\tm@shift@end@x,\tm@shift@end@y)}]#3-\tm@i);
        \else
          \coordinate (tm@line@aux1) at ([shift={(\tm@shift@start@x,\tm@shift@start@y)}]#3-\tm@i);
          \coordinate (tm@line@aux2) at ([shift={(\tm@shift@end@x,\tm@shift@end@y)}]#2-\tm@i);
        \fi
        \draw[decorate,decoration={calligraphic curved parenthesis,amplitude=\tm@amplitude,mirror},very thick,pen colour/.expanded=\tm@color]
          ([shift={(2mm,-1mm)}]tm@line@aux1) -- ([shift={(-2mm,-1mm)}]tm@line@aux2);
      \fi
    }
  \endgroup
}

\newlength\tm@line@tail@x
\newlength\tm@line@tail@y

\def\tm@line@init#1#2{
  \def\tm@note@min{-4}
  \def\tm@note@max{4}
  \def\tm@note@tail@min{52}
  \def\tm@note@tail@max{1}
  %
  \tm@note@getnote{#1}
  \path (#1-tail); \pgfgetlastxy{\tm@line@tail@x}{\tm@line@tail@y}
  \pgfmathsetmacro\tm@note@tail@min{min(scalar(\tm@line@tail@y/1mm),\tm@note@tail@min)}
  \pgfmathsetmacro\tm@note@tail@max{max(scalar(\tm@line@tail@y/1mm),\tm@note@tail@max)}
  \pgfmathparse{min(\tm@note@min,\tm@note@min@diff-1,\tm@note@tail@min)}
  \let\tm@note@min\pgfmathresult
  \pgfmathparse{max(\tm@note@max,\tm@note@max@diff+1,\tm@note@tail@max)}
  \let\tm@note@max\pgfmathresult
  \pgfmathsetmacro\tm@line@posone{\tm@note@pos}
  %
  \tm@note@getnote{#2}
  \path (#2-tail); \pgfgetlastxy{\tm@line@tail@x}{\tm@line@tail@y}
  \pgfmathsetmacro\tm@note@tail@min{min(scalar(\tm@line@tail@y/1mm),\tm@note@tail@min)}
  \pgfmathsetmacro\tm@note@tail@max{max(scalar(\tm@line@tail@y/1mm),\tm@note@tail@max)}
  \pgfmathparse{min(\tm@note@min,\tm@note@min@diff-1,\tm@note@tail@min)}
  \let\tm@note@min\pgfmathresult
  \pgfmathparse{max(\tm@note@max,\tm@note@max@diff+1,\tm@note@tail@max)}
  \let\tm@note@max\pgfmathresult
  \pgfmathsetmacro\tm@line@postwo{\tm@note@pos}
}

%%%%%%%%%%%%%%%%%%%%%%%%%%%%%%%%%%%
% CRESCENDO

% The hairpin
\newcommand\tmcrescendohairpin[3][]{
  \begingroup
    \tmset{#1}
    \tm@line@init{#2}{#3}
    \coordinate (tm@line@aux1) at ([xshift/.evaluated=-2mm+\tm@shift@start@x,yshift=\tm@shift@start@y]\tm@line@posone,\tm@note@min*.1-.5);
    \coordinate (tm@line@aux2) at ([xshift/.evaluated=2mm+\tm@shift@end@x,yshift=\tm@shift@end@y]\tm@line@postwo,\tm@note@min*.1-.5);
    \draw[color/.expanded=\tm@color] ([yshift={.5*\tm@crescdimsep}]tm@line@aux2) -- (tm@line@aux1) -- ([yshift={-.5*\tm@crescdimsep}]tm@line@aux2);
  \endgroup
}
% The line
\newcommand\tmcrescendoline[3][]{
  \begingroup
    \tmset{#1}
    \tm@line@init{#2}{#3}
    \coordinate (tm@line@aux1) at ([xshift/.evaluated=-2mm+\tm@shift@start@x,yshift=\tm@shift@start@y]\tm@line@posone,\tm@note@min*.1-.5);
    \coordinate (tm@line@aux2) at ([xshift/.evaluated=2mm+\tm@shift@end@x,yshift=\tm@shift@end@y]\tm@line@postwo,\tm@note@min*.1-.5);
    \begin{scope}[color/.expanded=\tm@color]
      \path (tm@line@aux1) node[above right,inner sep=0pt,font=\itshape] (tm@line@aux@node) {cresc.};
      \draw[dashed] (tm@line@aux@node.base east) -- (tm@line@aux2);
    \end{scope}
  \endgroup
}
\let\tmcrescendo\tmcrescendohairpin
% Custom coordinates
\newcommand\tmcrescendohairpincoordinate[3][]{
  \begingroup
    \tmset{#1}
    \draw[color/.expanded=\tm@color] 
      ([yshift={.5*\tm@crescdimsep},shift={(\tm@shift@end@x,\tm@shift@end@y)}]#3) -- 
      ([shift={(\tm@shift@start@x,\tm@shift@start@y)}]#2) -- 
      ([yshift={-.5*\tm@crescdimsep},shift={(\tm@shift@end@x,\tm@shift@end@y)}]#3);
  \endgroup
}
\newcommand\tmcrescendolinecoordinate[3][]{
  \begingroup
    \tmset{#1}
    \begin{scope}[color/.expanded=\tm@color]
      \path ([shift={(\tm@shift@start@x,\tm@shift@start@y)}]#2) node[above right,inner sep=0pt,font=\itshape] (tm@line@aux@node) {cresc.};
      \draw[dashed] (tm@line@aux@node.base east) -- ([shift={(\tm@shift@end@x,\tm@shift@end@y)}]#3);
    \end{scope}
  \endgroup
}
\let\tmcrescendocoordinate\tmcrescendohairpincoordinate

%%%%%%%%%%%%%%%%%%%%%%%%%%%%%%%%%%%
% DIMINUENDO

% The hairpin
\newcommand\tmdiminuendohairpin[3][]{
  \begingroup
    \tmset{#1}
    \tm@line@init{#2}{#3}
    \coordinate (tm@line@aux1) at ([xshift/.evaluated=-2mm+\tm@shift@start@x,yshift=\tm@shift@start@y]\tm@line@posone,\tm@note@min*.1-.5);
    \coordinate (tm@line@aux2) at ([xshift/.evaluated=2mm+\tm@shift@end@x,yshift=\tm@shift@end@y]\tm@line@postwo,\tm@note@min*.1-.5);
    \begin{scope}[color/.expanded=\tm@color]
      \draw ([yshift={.5*\tm@crescdimsep}]tm@line@aux1) -- (tm@line@aux2) -- ([yshift={-.5*\tm@crescdimsep}]tm@line@aux1);
    \end{scope}
  \endgroup
}
% The line
\newcommand\tmdiminuendoline[3][]{
  \begingroup
    \tmset{#1}
    \tm@line@init{#2}{#3}
    \coordinate (tm@line@aux1) at ([xshift/.evaluated=-2mm+\tm@shift@start@x,yshift=\tm@shift@start@y]\tm@line@posone,\tm@note@min*.1-.5);
    \coordinate (tm@line@aux2) at ([xshift/.evaluated=2mm+\tm@shift@end@x,yshift=\tm@shift@end@y]\tm@line@postwo,\tm@note@min*.1-.5);
    \begin{scope}[color/.expanded=\tm@color]
      \path (tm@line@aux1) node[above right,inner sep=0pt,font=\itshape] (tm@line@aux@node) {dim.};
      \draw[dashed] (tm@line@aux@node.base east) -- (tm@line@aux2);
    \end{scope}
  \endgroup
}
\let\tmdiminuendo\tmdiminuendohairpin
% Custom coordinates
\newcommand\tmdiminuendohairpincoordinate[3][]{
  \begingroup
    \tmset{#1}
    \draw[color/.expanded=\tm@color] 
      ([yshift={.5*\tm@crescdimsep},shift={(\tm@shift@start@x,\tm@shift@start@y)}]#2) -- 
      ([shift={(\tm@shift@end@x,\tm@shift@end@y)}]#3) -- 
      ([yshift={-.5*\tm@crescdimsep},shift={(\tm@shift@start@x,\tm@shift@start@y)}]#2);
  \endgroup
}
\newcommand\tmdiminuendolinecoordinate[3][]{
  \begingroup
    \tmset{#1}
    \begin{scope}[color/.expanded=\tm@color]
      \path ([shift={(\tm@shift@start@x,\tm@shift@start@y)}]#2) node[above right,inner sep=0pt,font=\itshape] (tm@line@aux@node) {dim.};
      \draw[dashed] (tm@line@aux@node.base east) -- ([shift={(\tm@shift@end@x,\tm@shift@end@y)}]#3);
    \end{scope}
  \endgroup
}
\let\tmdiminuendocoordinate\tmdiminuendohairpincoordinate

%%%%%%%%%%%%%%%%%%%%%%%%%%%%%%%%%%%
% VOLTA

\newcommand\tmvolta[4][]{
  \begingroup
    \tmset{#1}
    \tm@line@init{#2}{#3}
    \coordinate (tm@line@aux1) at ([xshift/.evaluated=-2mm+\tm@shift@start@x,yshift=\tm@shift@start@y]\tm@line@posone,\tm@note@max*.1+.3);
    \coordinate (tm@line@aux2) at ([xshift/.evaluated=2mm+\tm@shift@end@x,yshift=\tm@shift@end@y]\tm@line@postwo,\tm@note@max*.1+.3);
    \begin{scope}[color/.expanded=\tm@color]
      \node[anchor=west,font=\bfseries,inner sep=.2em] (tm@line@auxnode) at (tm@line@aux1) {#4.};
      \iftm@line@voltaunclosed
        \draw (tm@line@auxnode.south west) -- (tm@line@auxnode.north west) -- 
          (tm@line@auxnode.north west -| tm@line@aux2);
      \else
        \draw (tm@line@auxnode.south west) -- (tm@line@auxnode.north west) -- 
          (tm@line@auxnode.north west -| tm@line@aux2) -- 
          (tm@line@auxnode.south west -| tm@line@aux2);
      \fi
    \end{scope}
  \endgroup
}
% Custom coordinates
\newcommand\tmvoltacoordinate[4][]{
  \begingroup
    \tmset{#1}
    \coordinate (tm@line@aux1) at ([shift={(\tm@shift@start@x,\tm@shift@start@y)}]#2);
    \coordinate (tm@line@aux2) at ([shift={(\tm@shift@end@x,\tm@shift@end@y)}]#3);
    \begin{scope}[color/.expanded=\tm@color]
      \node[anchor=west,font=\bfseries,inner sep=.2em] (tm@line@auxnode) at (tm@line@aux1) {#4.};
      \iftm@line@voltaunclosed
        \draw (tm@line@auxnode.south west) -- (tm@line@auxnode.north west) -- 
          (tm@line@auxnode.north west -| tm@line@aux2);
      \else
        \draw (tm@line@auxnode.south west) -- (tm@line@auxnode.north west) -- 
          (tm@line@auxnode.north west -| tm@line@aux2) -- 
          (tm@line@auxnode.south west -| tm@line@aux2);
      \fi
    \end{scope}
  \endgroup
}

%%%%%%%%%%%%%%%%%%%%%%%%%%%%%%%%%%%
% OCTAVE LINES

\def\tm@string@eva{8va}
\def\tm@string@evb{8vb}
\def\tm@string@fma{15ma}
\def\tm@string@fmb{15mb}
\def\tm@octave#1#2#3{
  \def\tm@string@aux{#3}
  \ifx\tm@string@aux\tm@string@eva
    \coordinate (tm@octave@aux1) at ([xshift=6mm]#1);
    \coordinate (tm@octave@aux2) at (#2);
    \begin{scope}[color/.expanded=\tm@color]
      \pic[color/.expanded=\tm@color] at (tm@octave@aux1) {tm-8va};
      \draw[dashed] (tm@octave@aux2) -- (tm@octave@aux1);
      \draw (tm@octave@aux2) -- ++ (0,-.3);
    \end{scope}
  \fi
  \ifx\tm@string@aux\tm@string@evb
    \coordinate (tm@octave@aux1) at ([xshift=2.5mm]#1);
    \coordinate (tm@octave@aux2) at (#2);
    \begin{scope}[color/.expanded=\tm@color]
      \pic[color/.expanded=\tm@color] at (tm@octave@aux1) {tm-8vb};
      \draw[dashed] (tm@octave@aux2) -- (tm@octave@aux1);
      \draw (tm@octave@aux2) -- ++ (0,.3);
    \end{scope}
  \fi
  \ifx\tm@string@aux\tm@string@fma
    \coordinate (tm@octave@aux1) at ([xshift=8mm]#1);
    \coordinate (tm@octave@aux2) at (#2);
    \begin{scope}[color/.expanded=\tm@color]
      \pic[color/.expanded=\tm@color] at (tm@octave@aux1) {tm-15ma};
      \draw[dashed] (tm@octave@aux2) -- (tm@octave@aux1);
      \draw (tm@octave@aux2) -- ++ (0,-.3);
    \end{scope}
  \fi
  \ifx\tm@string@aux\tm@string@fmb
    \coordinate (tm@octave@aux1) at ([xshift=3.5mm]#1);
    \coordinate (tm@octave@aux2) at (#2);
    \begin{scope}[color/.expanded=\tm@color]
      \pic[color/.expanded=\tm@color] at (tm@octave@aux1) {tm-15mb};
      \draw[dashed] (tm@octave@aux2) -- (tm@octave@aux1);
      \draw (tm@octave@aux2) -- ++ (0,.3);
    \end{scope}
  \fi
}
\newcommand\tmoctave[4][]{
  \begingroup
    \tmset{#1}
    \tm@line@init{#2}{#3}
    \def\tm@string@aux{#4}
    \ifx\tm@string@aux\tm@string@eva
      \coordinate (tm@line@aux1) at ([xshift/.evaluated=-3mm+\tm@shift@start@x,yshift=\tm@shift@start@y]\tm@line@posone,\tm@note@max*.1+.5);
      \coordinate (tm@line@aux2) at ([xshift/.evaluated=2mm+\tm@shift@end@x,yshift=\tm@shift@end@y]\tm@line@postwo,\tm@note@max*.1+.5);
      \tm@octave{tm@line@aux1}{tm@line@aux2}{8va}
    \fi
    \ifx\tm@string@aux\tm@string@evb
      \coordinate (tm@line@aux1) at ([xshift/.evaluated=-3mm+\tm@shift@start@x,yshift=\tm@shift@start@y]\tm@line@posone,\tm@note@min*.1-.5);
      \coordinate (tm@line@aux2) at ([xshift/.evaluated=2mm+\tm@shift@end@x,yshift=\tm@shift@end@y]\tm@line@postwo,\tm@note@min*.1-.5);
      \tm@octave{tm@line@aux1}{tm@line@aux2}{8vb}
    \fi
    \ifx\tm@string@aux\tm@string@fma
      \coordinate (tm@line@aux1) at ([xshift/.evaluated=-3mm+\tm@shift@start@x,yshift=\tm@shift@start@y]\tm@line@posone,\tm@note@max*.1+.5);
      \coordinate (tm@line@aux2) at ([xshift/.evaluated=2mm+\tm@shift@end@x,yshift=\tm@shift@end@y]\tm@line@postwo,\tm@note@max*.1+.5);
      \tm@octave{tm@line@aux1}{tm@line@aux2}{15ma}
    \fi
    \ifx\tm@string@aux\tm@string@fmb
      \coordinate (tm@line@aux1) at ([xshift/.evaluated=-3mm+\tm@shift@start@x,yshift=\tm@shift@start@y]\tm@line@posone,\tm@note@min*.1-.5);
      \coordinate (tm@line@aux2) at ([xshift/.evaluated=2mm+\tm@shift@end@x,yshift=\tm@shift@end@y]\tm@line@postwo,\tm@note@min*.1-.5);
      \tm@octave{tm@line@aux1}{tm@line@aux2}{15mb}
    \fi
  \endgroup
}
% Custom coordinates
\newcommand\tmoctavecoordinate[4][]{
  \begingroup
    \tmset{#1}
    \path ([shift={(\tm@shift@start@x,\tm@shift@start@y)}]#2) coordinate (tm@line@aux1) 
          ([shift={(\tm@shift@end@x,\tm@shift@end@y)}]#3) coordinate (tm@line@aux2);
    \tm@octave{tm@line@aux1}{tm@line@aux2}{#4}
  \endgroup
}

%%%%%%%%%%%%%%%%%%%%%%%%%%%%%%%%%%%
% PEDAL LINES

\newcommand\tmpedal[3][]{
  \begingroup
    \tmset{#1}
    \tm@line@init{#2}{#3}
    \def\tm@string@aux{#2}
    \def\tm@string@aux@{#3}
    \ifx\tm@string@aux\tm@string@aux@
      \pic[color/.expanded=\tm@color] at ([xshift/.evaluated=3mm+\tm@shift@start@x,yshift=\tm@shift@start@y]\tm@line@posone,\tm@note@min*.1-.5) {tm-ped};
    \else
      \coordinate (tm@line@aux1) at ([xshift/.evaluated=3mm+\tm@shift@start@x,yshift=\tm@shift@start@y]\tm@line@posone,\tm@note@min*.1-.5);
      \coordinate (tm@line@aux2) at ([xshift/.evaluated=2mm+\tm@shift@end@x,yshift=\tm@shift@end@y]\tm@line@postwo,\tm@note@min*.1-.5);
      \begin{scope}[color/.expanded=\tm@color]
        \pic[color/.expanded=\tm@color] at (tm@line@aux1) {tm-ped};
        \draw (tm@line@aux1) -- (tm@line@aux2) -- ++ (0,.2);
      \end{scope}
    \fi
  \endgroup
}
\newcommand\tmpedalstar[3][]{
  \begingroup
    \tmset{#1}
    \tm@line@init{#2}{#3}
    \def\tm@string@aux{#2}
    \def\tm@string@aux@{#3}
    \ifx\tm@string@aux\tm@string@aux@
      \pic[color/.expanded=\tm@color] at ([xshift/.evaluated=3mm+\tm@shift@start@x,yshift=\tm@shift@start@y]\tm@line@posone,\tm@note@min*.1-.5) {tm-ped};
      \pic[color/.expanded=\tm@color] at ([xshift/.evaluated=3mm+\tm@shift@start@x,yshift=\tm@shift@start@y]\tm@line@posone,\tm@note@min*.1-.5) {tm-ped-star};
    \else
      \coordinate (tm@line@aux1) at ([xshift/.evaluated=3mm+\tm@shift@start@x,yshift=\tm@shift@start@y]\tm@line@posone,\tm@note@min*.1-.5);
      \coordinate (tm@line@aux2) at ([xshift=\tm@shift@end@x,yshift=\tm@shift@end@y]\tm@line@postwo,\tm@note@min*.1-.5);
      \begin{scope}[color/.expanded=\tm@color]
        \pic[color/.expanded=\tm@color] at (tm@line@aux1) {tm-ped};
        \draw (tm@line@aux1) -- (tm@line@aux2) pic[color/.expanded=\tm@color] {tm-ped-star};
      \end{scope}
    \fi
  \endgroup
}
% Custom coordinates
\newcommand\tmpedalcoordinate[3][]{
  \begingroup
    \tmset{#1}
    \def\tm@string@aux{#2}
    \def\tm@string@aux@{#3}
    \ifx\tm@string@aux\tm@string@aux@
      \pic[color/.expanded=\tm@color] at ([xshift=\tm@shift@start@x,yshift=\tm@shift@start@y]#2) {tm-ped};
    \else
      \coordinate (tm@line@aux1) at ([xshift=\tm@shift@start@x,yshift=\tm@shift@start@y]#2);
      \coordinate (tm@line@aux2) at ([xshift=\tm@shift@end@x,yshift=\tm@shift@end@y]#3);
      \begin{scope}[color/.expanded=\tm@color]
        \pic[color/.expanded=\tm@color] at (tm@line@aux1) {tm-ped};
        \draw (tm@line@aux1) -- (tm@line@aux2) -- ++ (0,.2);
      \end{scope}
    \fi
  \endgroup
}
\newcommand\tmpedalstarcoordinate[3][]{
  \begingroup
    \tmset{#1}
    \def\tm@string@aux{#2}
    \def\tm@string@aux@{#3}
    \ifx\tm@string@aux\tm@string@aux@
      \pic[color/.expanded=\tm@color] at ([xshift=\tm@shift@start@x,yshift=\tm@shift@start@y]#2) {tm-ped};
      \pic[color/.expanded=\tm@color] at ([xshift=\tm@shift@start@x,yshift=\tm@shift@start@y]#3) {tm-ped-star};
    \else
      \coordinate (tm@line@aux1) at ([xshift=\tm@shift@start@x,yshift=\tm@shift@start@y]#2);
      \coordinate (tm@line@aux2) at ([xshift=\tm@shift@end@x,yshift=\tm@shift@end@y]#3);
      \begin{scope}[color/.expanded=\tm@color]
        \pic[color/.expanded=\tm@color] at (tm@line@aux1) {tm-ped};
        \draw (tm@line@aux1) -- (tm@line@aux2) pic[color/.expanded=\tm@color] {tm-ped-star};
      \end{scope}
    \fi
  \endgroup
}


%%%%%%%%%%%%%%%%%%%%%%%%%%%%
% TRILL LINE

% Not intended for general use
\pgfdeclaredecoration{tm@trillline}{initial}{
  \state{initial}[width=.22cm]{
    \fill[color/.expanded=\tm@color] (0,0) sin ++ (.055,.04) cos ++ (.055,-.04) sin ++ (.055,-.04) cos ++ (.055,.04)
      -- ++ (60:.08) sin ++ (-.055,-.04) cos ++ (-.055,.04) sin ++ (-.055,.04) cos ++ (-.055,-.04) -- cycle;
  }
  \state{final}{}
}

\newcommand\tmtrillline[3][]{
  \begingroup
    \tmset{#1}
    \tm@line@init{#2}{#3}
    \coordinate (tm@line@aux1) at ([xshift/.evaluated=\tm@shift@start@x,yshift=\tm@shift@start@y]\tm@line@posone,\tm@note@max*.1+.5);
    \coordinate (tm@line@aux2) at ([xshift/.evaluated=-1.5mm+\tm@shift@end@x,yshift=\tm@shift@end@y]\tm@line@postwo,\tm@note@max*.1+.5);
    \draw[decorate,decoration=tm@trillline] ([shift={(2mm,0mm)}]tm@line@aux1) -- (tm@line@aux2);
    \pic[color/.expanded=\tm@color] at (tm@line@aux1) {tm-trill};
  \endgroup
}
\newcommand\tmtrilllinecoordinate[3][]{
  \begingroup
    \tmset{#1}
    \coordinate (tm@line@aux1) at ([xshift=\tm@shift@start@x,yshift=\tm@shift@start@y]#2);
    \coordinate (tm@line@aux2) at ([xshift=\tm@shift@end@x,yshift=\tm@shift@end@y]#3);
    \draw[decorate,decoration=tm@trillline] ([shift={(2mm,0mm)}]tm@line@aux1) -- (tm@line@aux2);
    \pic[color/.expanded=\tm@color] at (tm@line@aux1) {tm-trill};
  \endgroup
}