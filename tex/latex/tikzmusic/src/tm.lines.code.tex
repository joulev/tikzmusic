%%%%%%%%%%%%%%%%%%%%%%%%%%%%%%%%%%%
% SLUR

\def\tmslur{\@ifstar\tm@@slur\tm@slur}
% Below
\NewDocumentCommand\tm@slur{ O{2.5mm} O{0,0} m O{0,0} m }{
  \tm@note@getnote{#3}\coordinate (tm@line@aux1) at (\tm@note@pos,\tm@note@min@diff*.1-.2);
  \tm@note@getnote{#5}\coordinate (tm@line@aux2) at (\tm@note@pos,\tm@note@min@diff*.1-.2);
  \draw[decorate,decoration={calligraphic curved parenthesis,amplitude=#1,mirror},very thick,color/.expanded=\tmcolor]
    ([shift={(#2)}]tm@line@aux1) -- ([shift={(#4)}]tm@line@aux2);
}
% Above
\NewDocumentCommand\tm@@slur{ O{2.5mm} O{0,0} m O{0,0} m }{
  \tm@note@getnote{#3}\coordinate (tm@line@aux1) at (\tm@note@pos,\tm@note@max@diff*.1+.2);
  \tm@note@getnote{#5}\coordinate (tm@line@aux2) at (\tm@note@pos,\tm@note@max@diff*.1+.2);
  \draw[decorate,decoration={calligraphic curved parenthesis,amplitude=#1},very thick,color/.expanded=\tmcolor]
    ([shift={(#2)}]tm@line@aux1) -- ([shift={(#4)}]tm@line@aux2);
}


\newlength\tm@line@tail@x
\newlength\tm@line@tail@y

\def\tm@line@init#1#2{
  \def\tm@note@min{-4}
  \def\tm@note@max{4}
  \def\tm@note@tail@min{52}
  \def\tm@note@tail@max{1}
  %
  \tm@note@getnote{#1}
  \path (#1-tail); \pgfgetlastxy{\tm@line@tail@x}{\tm@line@tail@y}
  \pgfmathsetmacro\tm@note@tail@min{min(scalar(\tm@line@tail@y/1mm),\tm@note@tail@min)}
  \pgfmathsetmacro\tm@note@tail@max{max(scalar(\tm@line@tail@y/1mm),\tm@note@tail@max)}
  \pgfmathparse{min(\tm@note@min,\tm@note@min@diff,\tm@note@tail@min)}
  \let\tm@note@min\pgfmathresult
  \pgfmathparse{max(\tm@note@max,\tm@note@max@diff,\tm@note@tail@max)}
  \let\tm@note@max\pgfmathresult
  \pgfmathsetmacro\tm@line@posone{\tm@note@pos}
  %
  \tm@note@getnote{#2}
  \path (#2-tail); \pgfgetlastxy{\tm@line@tail@x}{\tm@line@tail@y}
  \pgfmathsetmacro\tm@note@tail@min{min(scalar(\tm@line@tail@y/1mm),\tm@note@tail@min)}
  \pgfmathsetmacro\tm@note@tail@max{max(scalar(\tm@line@tail@y/1mm),\tm@note@tail@max)}
  \pgfmathparse{min(\tm@note@min,\tm@note@min@diff,\tm@note@tail@min)}
  \let\tm@note@min\pgfmathresult
  \pgfmathparse{max(\tm@note@max,\tm@note@max@diff,\tm@note@tail@max)}
  \let\tm@note@max\pgfmathresult
  \pgfmathsetmacro\tm@line@postwo{\tm@note@pos}
}

%%%%%%%%%%%%%%%%%%%%%%%%%%%%%%%%%%%
% CRESCENDO

\def\tmcrescendo{\@ifstar\tm@@crescendo\tm@crescendo}
% The hairpin
\NewDocumentCommand\tm@crescendo{ O{0mm} O{3mm} m m }{
  \tm@line@init{#3}{#4}
  \coordinate (tm@line@aux1) at ([xshift=-2mm,yshift=#1]\tm@line@posone,\tm@note@min*.1-.5);
  \coordinate (tm@line@aux2) at ([xshift=2mm,yshift=#1]\tm@line@postwo,\tm@note@min*.1-.5);
  \draw[color/.expanded=\tmcolor] ([yshift={.5*#2}]tm@line@aux2) -- (tm@line@aux1) -- ([yshift={-.5*#2}]tm@line@aux2);
}
% The line
\newcommand\tm@@crescendo[3][0mm]{
  \tm@line@init{#2}{#3}
  \coordinate (tm@line@aux1) at ([xshift=-2mm,yshift=#1]\tm@line@posone,\tm@note@min*.1-.5);
  \coordinate (tm@line@aux2) at ([xshift=2mm,yshift=#1]\tm@line@postwo,\tm@note@min*.1-.5);
  \begin{scope}[color/.expanded=\tmcolor]
    \path (tm@line@aux1) node[above right,inner sep=0pt,font=\itshape] (tm@line@aux@node) {cresc.};
    \draw[dashed] (tm@line@aux@node.base east) -- (tm@line@aux2);
  \end{scope}
}
% Custom coordinates
\def\tmcrescendoline{\@ifstar\tm@@crescendoline\tm@crescendoline}
\newcommand\tm@crescendoline[3][3mm]{
  \draw[color/.expanded=\tmcolor] ([yshift={.5*#1}]#3) -- (#2) -- ([yshift={-.5*#1}]#3);
}
\def\tm@@crescendoline#1#2{
  \begin{scope}[color/.expanded=\tmcolor]
    \path (#1) node[above right,inner sep=0pt,font=\itshape] (tm@line@aux@node) {cresc.};
    \draw[dashed] (tm@line@aux@node.base east) -- (#2);
  \end{scope}
}

%%%%%%%%%%%%%%%%%%%%%%%%%%%%%%%%%%%
% DIMINUENDO

\def\tmdiminuendo{\@ifstar\tm@@diminuendo\tm@diminuendo}
% The hairpin
\NewDocumentCommand\tm@diminuendo{ O{0mm} O{3mm} m m }{
  \tm@line@init{#3}{#4}
  \coordinate (tm@line@aux1) at ([xshift=-2mm,yshift=#1]\tm@line@posone,\tm@note@min*.1-.5);
  \coordinate (tm@line@aux2) at ([xshift=2mm,yshift=#1]\tm@line@postwo,\tm@note@min*.1-.5);
  \begin{scope}[color/.expanded=\tmcolor]
    \draw ([yshift={.5*#2}]tm@line@aux1) -- (tm@line@aux2) -- ([yshift={-.5*#2}]tm@line@aux1);
  \end{scope}
}
% The line
\newcommand\tm@@diminuendo[3][0mm]{
  \tm@line@init{#2}{#3}
  \coordinate (tm@line@aux1) at ([xshift=-2mm,yshift=#1]\tm@line@posone,\tm@note@min*.1-.5);
  \coordinate (tm@line@aux2) at ([xshift=2mm,yshift=#1]\tm@line@postwo,\tm@note@min*.1-.5);
  \begin{scope}[color/.expanded=\tmcolor]
    \path (tm@line@aux1) node[above right,inner sep=0pt,font=\itshape] (tm@line@aux@node) {dim.};
    \draw[dashed] (tm@line@aux@node.base east) -- (tm@line@aux2);
  \end{scope}
}
% Custom coordinates
\def\tmdiminuendoline{\@ifstar\tm@@diminuendoline\tm@diminuendoline}
\newcommand\tm@diminuendoline[3][3mm]{
  \draw[color/.expanded=\tmcolor] ([yshift={.5*#1}]#2) -- (#3) -- ([yshift={-.5*#1}]#2);
}
\def\tm@@diminuendoline#1#2{
  \begin{scope}[color/.expanded=\tmcolor]
    \path (#1) node[above right,inner sep=0pt,font=\itshape] (tm@line@aux@node) {dim.};
    \draw[dashed] (tm@line@aux@node.base east) -- (#2);
  \end{scope}
}

%%%%%%%%%%%%%%%%%%%%%%%%%%%%%%%%%%%
% VOLTA

\def\tmvolta{\@ifstar\tm@@volta\tm@volta}
% Closed
\newcommand\tm@volta[4][0mm]{
  \tm@line@init{#2}{#3}
  \coordinate (tm@line@aux1) at ([xshift=-2mm,yshift=#1]\tm@line@posone,\tm@note@max*.1+.3);
  \coordinate (tm@line@aux2) at ([xshift=2mm,yshift=#1]\tm@line@postwo,\tm@note@max*.1+.3);
  \begin{scope}[color/.expanded=\tmcolor]
    \node[anchor=west,font=\bfseries,inner sep=.2em] (tm@line@auxnode) at (tm@line@aux1) {#4.};
    \draw (tm@line@auxnode.south west) -- (tm@line@auxnode.north west) -- 
      (tm@line@auxnode.north west -| tm@line@aux2) -- 
      (tm@line@auxnode.south west -| tm@line@aux2);
  \end{scope}
}
% Unclosed
\newcommand\tm@@volta[4][0mm]{
  \tm@line@init{#2}{#3}
  \coordinate (tm@line@aux1) at ([xshift=-2mm,yshift=#1]\tm@line@posone,\tm@note@max*.1+.3);
  \coordinate (tm@line@aux2) at ([xshift=2mm,yshift=#1]\tm@line@postwo,\tm@note@max*.1+.3);
  \begin{scope}[color/.expanded=\tmcolor]
    \node[anchor=west,font=\bfseries,inner sep=.2em] (tm@line@auxnode) at (tm@line@aux1) {#4.};
    \draw (tm@line@auxnode.south west) -- (tm@line@auxnode.north west) -- 
      (tm@line@auxnode.north west -| tm@line@aux2);
  \end{scope}
}
% Custom coordinates
\def\tmvoltaline{\@ifstar\tm@@voltaline\tm@voltaline}
% \tm@voltaline{y}{x1}{x2}{n}
\def\tm@voltaline#1#2#3#4{
  \coordinate (tm@line@aux1) at (#2,#1);
  \coordinate (tm@line@aux2) at (#3,#1);
  \begin{scope}[color/.expanded=\tmcolor]
    \node[anchor=west,font=\bfseries,inner sep=.2em] (tm@line@auxnode) at (tm@line@aux1) {#4.};
    \draw (tm@line@auxnode.south west) -- (tm@line@auxnode.north west) -- 
      (tm@line@auxnode.north west -| tm@line@aux2) -- 
      (tm@line@auxnode.south west -| tm@line@aux2);
  \end{scope}
}
\def\tm@@voltaline#1#2#3#4{
  \coordinate (tm@line@aux1) at (#2,#1);
  \coordinate (tm@line@aux2) at (#3,#1);
  \begin{scope}[color/.expanded=\tmcolor]
    \node[anchor=west,font=\bfseries,inner sep=.2em] (tm@line@auxnode) at (tm@line@aux1) {#4.};
    \draw (tm@line@auxnode.south west) -- (tm@line@auxnode.north west) -- 
      (tm@line@auxnode.north west -| tm@line@aux2);
  \end{scope}
}

%%%%%%%%%%%%%%%%%%%%%%%%%%%%%%%%%%%
% OCTAVE LINES

\def\tm@octave#1#2#3{
  \ifstrequal{#3}{8va}{
    \coordinate (tm@octave@aux1) at ([xshift=6mm]#1);
    \coordinate (tm@octave@aux2) at (#2);
    \begin{scope}[color/.expanded=\tmcolor]
      \pic[color/.expanded=\tmcolor] at (tm@octave@aux1) {tm-8va};
      \draw[dashed] (tm@octave@aux2) -- (tm@octave@aux1);
      \draw (tm@octave@aux2) -- ++ (0,-.3);
    \end{scope}
  }{}
  \ifstrequal{#3}{8vb}{
    \coordinate (tm@octave@aux1) at ([xshift=2.5mm]#1);
    \coordinate (tm@octave@aux2) at (#2);
    \begin{scope}[color/.expanded=\tmcolor]
      \pic[color/.expanded=\tmcolor] at (tm@octave@aux1) {tm-8vb};
      \draw[dashed] (tm@octave@aux2) -- (tm@octave@aux1);
      \draw (tm@octave@aux2) -- ++ (0,.3);
    \end{scope}
  }{}
  \ifstrequal{#3}{15ma}{
    \coordinate (tm@octave@aux1) at ([xshift=8mm]#1);
    \coordinate (tm@octave@aux2) at (#2);
    \begin{scope}[color/.expanded=\tmcolor]
      \pic[color/.expanded=\tmcolor] at (tm@octave@aux1) {tm-15ma};
      \draw[dashed] (tm@octave@aux2) -- (tm@octave@aux1);
      \draw (tm@octave@aux2) -- ++ (0,-.3);
    \end{scope}
  }{}
  \ifstrequal{#3}{15mb}{
    \coordinate (tm@octave@aux1) at ([xshift=3.5mm]#1);
    \coordinate (tm@octave@aux2) at (#2);
    \begin{scope}[color/.expanded=\tmcolor]
      \pic[color/.expanded=\tmcolor] at (tm@octave@aux1) {tm-15mb};
      \draw[dashed] (tm@octave@aux2) -- (tm@octave@aux1);
      \draw (tm@octave@aux2) -- ++ (0,.3);
    \end{scope}
  }{}
}
\newcommand\tmoctave[4][0mm]{
  \begin{scope}[yshift=#1]
    \tm@line@init{#2}{#3}
    \ifstrequal{#4}{8va}{
      \coordinate (tm@line@aux1) at ([xshift=-3mm]\tm@line@posone,\tm@note@max*.1+.5);
      \coordinate (tm@line@aux2) at ([xshift=2mm]\tm@line@postwo,\tm@note@max*.1+.5);
      \tm@octave{tm@line@aux1}{tm@line@aux2}{8va}
    }{}
    \ifstrequal{#4}{8vb}{
      \coordinate (tm@line@aux1) at ([xshift=-3mm]\tm@line@posone,\tm@note@min*.1-.5);
      \coordinate (tm@line@aux2) at ([xshift=2mm]\tm@line@postwo,\tm@note@min*.1-.5);
      \tm@octave{tm@line@aux1}{tm@line@aux2}{8vb}
    }{}
    \ifstrequal{#4}{15ma}{
      \coordinate (tm@line@aux1) at ([xshift=-3mm]\tm@line@posone,\tm@note@max*.1+.5);
      \coordinate (tm@line@aux2) at ([xshift=2mm]\tm@line@postwo,\tm@note@max*.1+.5);
      \tm@octave{tm@line@aux1}{tm@line@aux2}{15ma}
    }{}
    \ifstrequal{#4}{15mb}{
      \coordinate (tm@line@aux1) at ([xshift=-3mm]\tm@line@posone,\tm@note@min*.1-.5);
      \coordinate (tm@line@aux2) at ([xshift=2mm]\tm@line@postwo,\tm@note@min*.1-.5);
      \tm@octave{tm@line@aux1}{tm@line@aux2}{15mb}
    }{}
  \end{scope}
}
% Custom coordinates
\def\tmoctaveline#1#2#3{
  \path (#1) coordinate (tm@line@aux1) (#2) coordinate (tm@line@aux2);
  \tm@octave{tm@line@aux1}{tm@line@aux2}{#3}
}