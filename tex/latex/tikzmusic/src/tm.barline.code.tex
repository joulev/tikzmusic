%%%%%%%%%%%%%%%%%%%%%%%%%%%%%%%%%%%
% NORMAL BAR LINES
\def\tmbarline{\@ifstar\tm@@barline\tm@barline}
\newcommand\tm@@barline[4][]{%
  \begin{tikzpicture}[remember picture,overlay]
    \tmset{#1}
    \coordinate (x) at ([shift={(#2,2*0.2)}]#3-start);
    \coordinate (y) at ([shift={(#2,-2*0.2)}]#4-start);
    \draw[color/.expanded=\tm@color] (x) -- (y);
  \end{tikzpicture}%
}
\newcommand\tm@barline[3][]{%
  \begin{tikzpicture}[remember picture,overlay]
    \tmset{#1}
    \coordinate (x) at ([shift={(#2,2*0.2)}]#3-start);
    \draw[color/.expanded=\tm@color] (x) -- ++ (0,-4*0.2);
  \end{tikzpicture}%
}
\newcommand\tmbarlineendline[3][]{\tm@@barline[#1]{\linewidth-\pgflinewidth}{#2}{#3}}
\newcommand\tmbarlineinline[2][]{
  \begingroup\tmset{#1}\foreach \tm@i in {#2} {\draw[color/.expanded=\tm@color] (\tm@i,2*0.2) -- ++ (0,-4*0.2);}\endgroup
}



%%%%%%%%%%%%%%%%%%%%%%%%%%%%%%%%%%%
% DOTTED BAR LINES
\def\tmdottedbarline{\@ifstar\tm@@dottedbarline\tm@dottedbarline}
\newcommand\tm@@dottedbarline[4][]{%
  \begin{tikzpicture}[remember picture,overlay]
    \tmset{#1}
    \coordinate (x) at ([shift={(#2,2*0.2)}]#3-start);
    \coordinate (y) at ([shift={(#2,-2*0.2)}]#4-start);
    \draw[dotted,color/.expanded=\tm@color] (x) -- (y);
  \end{tikzpicture}%
}
\newcommand\tm@dottedbarline[3][]{%
  \begin{tikzpicture}[remember picture,overlay]
    \tmset{#1}
    \coordinate (x) at ([shift={(#2,2*0.2)}]#3-start);
    \draw[dotted,color/.expanded=\tm@color] (x) -- ++ (0,-4*0.2);
  \end{tikzpicture}%
}
\newcommand\tmdottedbarlineendline[3][]{\tm@@dottedbarline[#1]{\linewidth-\pgflinewidth}{#2}{#3}}
\newcommand\tmdottedbarlineinline[2][]{
  \begingroup\tmset{#1}\foreach \tm@i in {#2} {\draw[dotted,color/.expanded=\tm@color] (\tm@i,2*0.2) -- ++ (0,-4*0.2);}\endgroup
}


%%%%%%%%%%%%%%%%%%%%%%%%%%%%%%%%%%%
% DOUBLE BAR LINES
\def\tmdoublebarline{\@ifstar\tm@@doublebarline\tm@doublebarline}
\newcommand\tm@@doublebarline[4][]{%
  \begin{tikzpicture}[remember picture,overlay]
    \tmset{#1}
    \coordinate (x) at ([shift={(#2,2*0.2)}]#3-start);
    \coordinate (y) at ([shift={(#2,-2*0.2)}]#4-start);
    \draw[color/.expanded=\tm@color] ([xshift=-.5mm]x) -- ([xshift=-.5mm]y) ([xshift=.5mm]x) -- ([xshift=.5mm]y);
  \end{tikzpicture}%
}
\newcommand\tm@doublebarline[3][]{%
  \begin{tikzpicture}[remember picture,overlay]
    \tmset{#1}
    \coordinate (x) at ([shift={(#2,2*0.2)}]#3-start);
    \draw[color/.expanded=\tm@color] ([xshift=-.5mm]x) -- ++ (0,-4*0.2) ([xshift=.5mm]x) -- ++ (0,-4*0.2);
  \end{tikzpicture}%
}
\newcommand\tmdoublebarlineendline[3][]{\tm@@doublebarline[#1]{\linewidth-\pgflinewidth-.5mm}{#2}{#3}}
\newcommand\tmdoublebarlineinline[2][]{
  \begingroup\tmset{#1}\foreach \tm@i in {#2} {
    \draw[color/.expanded=\tm@color] 
      ([xshift=-.5mm]\tm@i,2*0.2) -- ++ (0,-4*0.2) ([xshift=.5mm] \tm@i,2*0.2) -- ++ (0,-4*0.2);
  }\endgroup
}



%%%%%%%%%%%%%%%%%%%%%%%%%%%%%%%%%%%
% FINAL BAR LINES
\def\tmfinalbarline{\@ifstar\tm@@finalbarline\tm@finalbarline}
\newcommand\tm@@finalbarline[4][]{%
  \begin{tikzpicture}[remember picture,overlay]
    \tmset{#1}
    \coordinate (x) at ([shift={(#2,2*0.2)}]#3-start);
    \coordinate (y) at ([shift={(#2,-2*0.2)}]#4-start);
    \scope[color/.expanded=\tm@color]
      \draw ([xshift=-.5mm]x) -- ([xshift=-.5mm]y);
      \draw[line width=.6mm] ([xshift=.5mm]x) -- ([xshift=.5mm]y);
    \endscope
  \end{tikzpicture}%
}
\newcommand\tm@finalbarline[3][]{%
  \begin{tikzpicture}[remember picture,overlay]
    \tmset{#1}
    \coordinate (x) at ([shift={(#2,2*0.2)}]#3-start);
    \scope[color/.expanded=\tm@color]
      \draw ([xshift=-.5mm]x) -- ++ (0,-4*0.2);
      \draw[line width=.6mm] ([xshift=.5mm]x) -- ++ (0,-4*0.2);
    \endscope
  \end{tikzpicture}%
}
\newcommand\tmfinalbarlineendline[3][]{\tm@@finalbarline[#1]{\linewidth-\pgflinewidth-.5mm}{#2}{#3}}
\newcommand\tmfinalbarlineinline[2][]{
  \begingroup\tmset{#1}\foreach \tm@i in {#2} {
    \scope[color/.expanded=\tm@color]
      \draw ([xshift=-.5mm]\tm@i,2*0.2) -- ++ (0,-4*0.2); 
      \draw[line width=.6mm] ([xshift=.5mm] \tm@i,2*0.2) -- ++ (0,-4*0.2);
    \endscope
  }\endgroup
}



%%%%%%%%%%%%%%%%%%%%%%%%%%%%%%%%%%%
% START REPEAT BAR LINES
\def\tmstartrepeatbarline{\@ifstar\tm@@startrepeatbarline\tm@startrepeatbarline}
\newcommand\tm@@startrepeatbarline[5][]{%
  \begin{tikzpicture}[remember picture,overlay]
    \tmset{#1}
    \coordinate (x) at ([shift={(#2,2*0.2)}]#3-start);
    \coordinate (y) at ([shift={(#2,-2*0.2)}]#4-start);
    \scope[color/.expanded=\tm@color]
      \draw[line width=.6mm] ([xshift=-.5mm]x) -- ([xshift=-.5mm]y);
      \draw ([xshift=.5mm]x) -- ([xshift=.5mm]y);
      \foreach \tm@i in {#5} {
        \coordinate (x) at ([shift={(#2,0)}]\tm@i-start);
        \fill ([shift={(1.5mm,1mm)}]x) circle (.4mm);
        \fill ([shift={(1.5mm,-1mm)}]x) circle (.4mm);
      }
    \endscope
  \end{tikzpicture}%
}
\newcommand\tm@startrepeatbarline[3][]{%
  \begin{tikzpicture}[remember picture,overlay]
    \tmset{#1}
    \coordinate (x) at ([shift={(#2,2*0.2)}]#3-start);
    \scope[color/.expanded=\tm@color]
      \draw[line width=.6mm] ([xshift=-.5mm]x) -- ++ (0,-4*0.2);
      \draw ([xshift=.5mm]x) -- ++ (0,-4*0.2);
      \coordinate (x) at ([shift={(#2,0)}]#3-start);
      \fill ([shift={(1.5mm,1mm)}]x) circle (.4mm);
      \fill ([shift={(1.5mm,-1mm)}]x) circle (.4mm);
    \endscope
  \end{tikzpicture}%
}
\newcommand\tmstartrepeatbarlineinline[2][]{
  \begingroup\tmset{#1}\foreach \tm@i in {#2} {
    \scope[color/.expanded=\tm@color]
      \draw[line width=.6mm] ([xshift=-.5mm]\tm@i,2*0.2) -- ++ (0,-4*0.2); 
      \draw ([xshift=.5mm]\tm@i,2*0.2) -- ++ (0,-4*0.2);
      \fill ([shift={(1.5mm,1mm)}]\tm@i,0) circle (.4mm);
      \fill ([shift={(1.5mm,-1mm)}]\tm@i,0) circle (.4mm);
    \endscope
  }\endgroup
}



%%%%%%%%%%%%%%%%%%%%%%%%%%%%%%%%%%%
% END REPEAT BAR LINES
\def\tmendrepeatbarline{\@ifstar\tm@@endrepeatbarline\tm@endrepeatbarline}
\newcommand\tm@@endrepeatbarline[5][]{%
  \begin{tikzpicture}[remember picture,overlay]
    \tmset{#1}
    \coordinate (x) at ([shift={(#2,2*0.2)}]#3-start);
    \coordinate (y) at ([shift={(#2,-2*0.2)}]#4-start);
    \scope[color/.expanded=\tm@color]
      \draw ([xshift=-.5mm]x) -- ([xshift=-.5mm]y);
      \draw[line width=.6mm] ([xshift=.5mm]x) -- ([xshift=.5mm]y);
      \foreach \tm@i in {#5} {
        \coordinate (x) at ([shift={(#2,0)}]\tm@i-start);
        \fill ([shift={(-1.5mm,1mm)}]x) circle (.4mm);
        \fill ([shift={(-1.5mm,-1mm)}]x) circle (.4mm);
      }
    \endscope
  \end{tikzpicture}%
}
\newcommand\tm@endrepeatbarline[3][]{%
  \begin{tikzpicture}[remember picture,overlay]
    \tmset{#1}
    \coordinate (x) at ([shift={(#2,2*0.2)}]#3-start);
    \scope[color/.expanded=\tm@color]
      \draw ([xshift=-.5mm]x) -- ++ (0,-4*0.2);
      \draw[line width=.6mm] ([xshift=.5mm]x) -- ++ (0,-4*0.2);
      \coordinate (x) at ([shift={(#2,0)}]#3-start);
      \fill ([shift={(-1.5mm,1mm)}]x) circle (.4mm);
      \fill ([shift={(-1.5mm,-1mm)}]x) circle (.4mm);
    \endscope
  \end{tikzpicture}%
}
\newcommand\tmendrepeatbarlineendline[4][]{\tm@@endrepeatbarline[#1]{\linewidth-\pgflinewidth-.5mm}{#2}{#3}{#4}}
\newcommand\tmendrepeatbarlineinline[2][]{
  \begingroup\tmset{#1}\foreach \tm@i in {#2} {
    \scope[color/.expanded=\tm@color]
      \draw ([xshift=-.5mm]\tm@i,2*0.2) -- ++ (0,-4*0.2); 
      \draw[line width=.6mm] ([xshift=.5mm]\tm@i,2*0.2) -- ++ (0,-4*0.2);
      \fill ([shift={(-1.5mm,1mm)}]\tm@i,0) circle (.4mm);
      \fill ([shift={(-1.5mm,-1mm)}]\tm@i,0) circle (.4mm);
    \endscope
  }\endgroup
}



%%%%%%%%%%%%%%%%%%%%%%%%%%%%%%%%%%%
% END-START REPEAT BAR LINES
\def\tmendstartrepeatbarline{\@ifstar\tm@@endstartrepeatbarline\tm@endstartrepeatbarline}
\newcommand\tm@@endstartrepeatbarline[5][]{%
  \begin{tikzpicture}[remember picture,overlay]
    \tmset{#1}
    \coordinate (x) at ([shift={(#2,2*0.2)}]#3-start);
    \coordinate (y) at ([shift={(#2,-2*0.2)}]#4-start);
    \scope[color/.expanded=\tm@color]
      \draw ([xshift=-1mm]x) -- ([xshift=-1mm]y) ([xshift=1mm]x) -- ([xshift=1mm]y);
      \draw[line width=.6mm] (x) -- (y);
      \foreach \tm@i in {#5} {
        \coordinate (x) at ([shift={(#2,0)}]\tm@i-start);
        \fill ([shift={(-2mm,1mm)}]x) circle (.4mm);
        \fill ([shift={(-2mm,-1mm)}]x) circle (.4mm);
        \fill ([shift={(2mm,1mm)}]x) circle (.4mm);
        \fill ([shift={(2mm,-1mm)}]x) circle (.4mm);
      }
    \endscope
  \end{tikzpicture}%
}
\newcommand\tm@endstartrepeatbarline[3][]{%
  \begin{tikzpicture}[remember picture,overlay]
    \tmset{#1}
    \scope[color/.expanded=\tm@color]
      \coordinate (x) at ([shift={(#2,2*0.2)}]#3-start);
      \draw ([xshift=-1mm]x) -- ++ (0,-4*0.2) ([xshift=1mm]x) -- ++ (0,-4*0.2);
      \draw[line width=.6mm] (x) -- ++ (0,-4*0.2);
      \coordinate (x) at ([shift={(#2,0)}]#3-start);
      \fill ([shift={(-2mm,1mm)}]x) circle (.4mm);
      \fill ([shift={(-2mm,-1mm)}]x) circle (.4mm);
      \fill ([shift={(2mm,1mm)}]x) circle (.4mm);
      \fill ([shift={(2mm,-1mm)}]x) circle (.4mm);
    \endscope
  \end{tikzpicture}%
}
\newcommand\tmendstartrepeatbarlineinline[2][]{
  \begingroup\tmset{#1}\foreach \tm@i in {#2} {
    \scope[color/.expanded=\tm@color]
      \draw ([xshift=-1mm]\tm@i,2*0.2) -- ++ (0,-4*0.2) ([xshift=1mm]\tm@i,2*0.2) -- ++ (0,-4*0.2);
      \draw[line width=.6mm] (\tm@i,2*0.2) -- ++ (0,-4*0.2);
      \fill ([shift={(-2mm,1mm)}]\tm@i,0) circle (.4mm);
      \fill ([shift={(-2mm,-1mm)}]\tm@i,0) circle (.4mm);
      \fill ([shift={(2mm,1mm)}]\tm@i,0) circle (.4mm);
      \fill ([shift={(2mm,-1mm)}]\tm@i,0) circle (.4mm);
    \endscope
  }\endgroup
}



%%%%%%%%%%%%%%%%%%%%%%%%%%%%%%%%%%%
% ADDITIONAL
\def\tmbarlineloop{\@ifstar\tm@@barlineloop\tm@barlineloop}
\newcommand\tm@barlineloop[3][]{\foreach \tm@i in {#2} {\foreach \tm@j in {#3} {\tmbarline[#1]{\tm@i}{\tm@j}}}}
\newcommand\tm@@barlineloop[4][]{\foreach \tm@i in {#2} {\tmbarline*[#1]{\tm@i}{#3}{#4}}}