%%%%%%%%%%%%%%%%%%%%%%%%%%%%%%%%%%%
% NORMAL BAR LINES
\newcommand\tmbarline[4][]{%
  \begin{tikzpicture}[remember picture,overlay]
    \tmset{#1}
    \coordinate (x) at ([shift={(#2,2*0.2)}]#3-start);
    \coordinate (y) at ([shift={(#2,-2*0.2)}]#4-start);
    \draw[color/.expanded=\tm@color] (x) -- (y);
  \end{tikzpicture}%
}
\newcommand\tmbarlineendline[3][]{\tmbarline[#1]{\linewidth-\pgflinewidth}{#2}{#3}}
\newcommand\tmbarlineinline[2][]{
  \begingroup\tmset{#1}\foreach \tm@i in {#2} {\draw[color/.expanded=\tm@color] (\tm@i,2*0.2) -- ++ (0,-4*0.2);}\endgroup
}



%%%%%%%%%%%%%%%%%%%%%%%%%%%%%%%%%%%
% DOTTED BAR LINES
\newcommand\tmdottedbarline[4][]{%
  \begin{tikzpicture}[remember picture,overlay]
    \tmset{#1}
    \coordinate (x) at ([shift={(#2,2*0.2)}]#3-start);
    \coordinate (y) at ([shift={(#2,-2*0.2)}]#4-start);
    \draw[dotted,color/.expanded=\tm@color] (x) -- (y);
  \end{tikzpicture}%
}
\newcommand\tmdottedbarlineendline[3][]{\tmdottedbarline[#1]{\linewidth-\pgflinewidth}{#2}{#3}}
\newcommand\tmdottedbarlineinline[2][]{
  \begingroup\tmset{#1}\foreach \tm@i in {#2} {\draw[dotted,color/.expanded=\tm@color] (\tm@i,2*0.2) -- ++ (0,-4*0.2);}\endgroup
}


%%%%%%%%%%%%%%%%%%%%%%%%%%%%%%%%%%%
% DOUBLE BAR LINES
\newcommand\tmdoublebarline[4][]{%
  \begin{tikzpicture}[remember picture,overlay]
    \tmset{#1}
    \coordinate (x) at ([shift={(#2,2*0.2)}]#3-start);
    \coordinate (y) at ([shift={(#2,-2*0.2)}]#4-start);
    \draw[color/.expanded=\tm@color] ([xshift=-.5mm]x) -- ([xshift=-.5mm]y) ([xshift=.5mm]x) -- ([xshift=.5mm]y);
  \end{tikzpicture}%
}
\newcommand\tmdoublebarlineendline[3][]{\tmdoublebarline[#1]{\linewidth-\pgflinewidth-.5mm}{#2}{#3}}
\newcommand\tmdoublebarlineinline[2][]{
  \begingroup\tmset{#1}\foreach \tm@i in {#2} {
    \draw[color/.expanded=\tm@color] 
      ([xshift=-.5mm]\tm@i,2*0.2) -- ++ (0,-4*0.2) ([xshift=.5mm] \tm@i,2*0.2) -- ++ (0,-4*0.2);
  }\endgroup
}



%%%%%%%%%%%%%%%%%%%%%%%%%%%%%%%%%%%
% FINAL BAR LINES
\newcommand\tmfinalbarline[4][]{%
  \begin{tikzpicture}[remember picture,overlay]
    \tmset{#1}
    \coordinate (x) at ([shift={(#2,2*0.2)}]#3-start);
    \coordinate (y) at ([shift={(#2,-2*0.2)}]#4-start);
    \scope[color/.expanded=\tm@color]
      \draw ([xshift=-.5mm]x) -- ([xshift=-.5mm]y);
      \draw[line width=.6mm] ([xshift=.5mm]x) -- ([xshift=.5mm]y);
    \endscope
  \end{tikzpicture}%
}
\newcommand\tmfinalbarlineendline[3][]{\tmfinalbarline[#1]{\linewidth-\pgflinewidth-.5mm}{#2}{#3}}
\newcommand\tmfinalbarlineinline[2][]{
  \begingroup\tmset{#1}\foreach \tm@i in {#2} {
    \scope[color/.expanded=\tm@color]
      \draw ([xshift=-.5mm]\tm@i,2*0.2) -- ++ (0,-4*0.2); 
      \draw[line width=.6mm] ([xshift=.5mm] \tm@i,2*0.2) -- ++ (0,-4*0.2);
    \endscope
  }\endgroup
}



%%%%%%%%%%%%%%%%%%%%%%%%%%%%%%%%%%%
% START REPEAT BAR LINES
\newcommand\tmstartrepeatbarline[5][]{%
  \begin{tikzpicture}[remember picture,overlay]
    \tmset{#1}
    \coordinate (x) at ([shift={(#2,2*0.2)}]#3-start);
    \coordinate (y) at ([shift={(#2,-2*0.2)}]#4-start);
    \scope[color/.expanded=\tm@color]
      \draw[line width=.6mm] ([xshift=-.5mm]x) -- ([xshift=-.5mm]y);
      \draw ([xshift=.5mm]x) -- ([xshift=.5mm]y);
      \foreach \tm@i in {#5} {
        \coordinate (x) at ([shift={(#2,0)}]\tm@i-start);
        \fill ([shift={(1.5mm,1mm)}]x) circle (.4mm);
        \fill ([shift={(1.5mm,-1mm)}]x) circle (.4mm);
      }
    \endscope
  \end{tikzpicture}%
}
\newcommand\tmstartrepeatbarlineinline[2][]{
  \begingroup\tmset{#1}\foreach \tm@i in {#2} {
    \scope[color/.expanded=\tm@color]
      \draw[line width=.6mm] ([xshift=-.5mm]\tm@i,2*0.2) -- ++ (0,-4*0.2); 
      \draw ([xshift=.5mm]\tm@i,2*0.2) -- ++ (0,-4*0.2);
      \fill ([shift={(1.5mm,1mm)}]\tm@i,0) circle (.4mm);
      \fill ([shift={(1.5mm,-1mm)}]\tm@i,0) circle (.4mm);
    \endscope
  }\endgroup
}



%%%%%%%%%%%%%%%%%%%%%%%%%%%%%%%%%%%
% END REPEAT BAR LINES
\newcommand\tmendrepeatbarline[5][]{%
  \begin{tikzpicture}[remember picture,overlay]
    \tmset{#1}
    \coordinate (x) at ([shift={(#2,2*0.2)}]#3-start);
    \coordinate (y) at ([shift={(#2,-2*0.2)}]#4-start);
    \scope[color/.expanded=\tm@color]
      \draw ([xshift=-.5mm]x) -- ([xshift=-.5mm]y);
      \draw[line width=.6mm] ([xshift=.5mm]x) -- ([xshift=.5mm]y);
      \foreach \tm@i in {#5} {
        \coordinate (x) at ([shift={(#2,0)}]\tm@i-start);
        \fill ([shift={(-1.5mm,1mm)}]x) circle (.4mm);
        \fill ([shift={(-1.5mm,-1mm)}]x) circle (.4mm);
      }
    \endscope
  \end{tikzpicture}%
}
\newcommand\tmendrepeatbarlineendline[4][]{\tmendrepeatbarline[#1]{\linewidth-\pgflinewidth-.5mm}{#2}{#3}{#4}}
\newcommand\tmendrepeatbarlineinline[2][]{
  \begingroup\tmset{#1}\foreach \tm@i in {#2} {
    \scope[color/.expanded=\tm@color]
      \draw ([xshift=-.5mm]\tm@i,2*0.2) -- ++ (0,-4*0.2); 
      \draw[line width=.6mm] ([xshift=.5mm]\tm@i,2*0.2) -- ++ (0,-4*0.2);
      \fill ([shift={(-1.5mm,1mm)}]\tm@i,0) circle (.4mm);
      \fill ([shift={(-1.5mm,-1mm)}]\tm@i,0) circle (.4mm);
    \endscope
  }\endgroup
}



%%%%%%%%%%%%%%%%%%%%%%%%%%%%%%%%%%%
% END-START REPEAT BAR LINES
\newcommand\tmendstartrepeatbarline[5][]{%
  \begin{tikzpicture}[remember picture,overlay]
    \tmset{#1}
    \coordinate (x) at ([shift={(#2,2*0.2)}]#3-start);
    \coordinate (y) at ([shift={(#2,-2*0.2)}]#4-start);
    \scope[color/.expanded=\tm@color]
      \draw ([xshift=-1mm]x) -- ([xshift=-1mm]y) ([xshift=1mm]x) -- ([xshift=1mm]y);
      \draw[line width=.6mm] (x) -- (y);
      \foreach \tm@i in {#5} {
        \coordinate (x) at ([shift={(#2,0)}]\tm@i-start);
        \fill ([shift={(-2mm,1mm)}]x) circle (.4mm);
        \fill ([shift={(-2mm,-1mm)}]x) circle (.4mm);
        \fill ([shift={(2mm,1mm)}]x) circle (.4mm);
        \fill ([shift={(2mm,-1mm)}]x) circle (.4mm);
      }
    \endscope
  \end{tikzpicture}%
}
\newcommand\tmendstartrepeatbarlineinline[2][]{
  \begingroup\tmset{#1}\foreach \tm@i in {#2} {
    \scope[color/.expanded=\tm@color]
      \draw ([xshift=-1mm]\tm@i,2*0.2) -- ++ (0,-4*0.2) ([xshift=1mm]\tm@i,2*0.2) -- ++ (0,-4*0.2);
      \draw[line width=.6mm] (\tm@i,2*0.2) -- ++ (0,-4*0.2);
      \fill ([shift={(-2mm,1mm)}]\tm@i,0) circle (.4mm);
      \fill ([shift={(-2mm,-1mm)}]\tm@i,0) circle (.4mm);
      \fill ([shift={(2mm,1mm)}]\tm@i,0) circle (.4mm);
      \fill ([shift={(2mm,-1mm)}]\tm@i,0) circle (.4mm);
    \endscope
  }\endgroup
}



%%%%%%%%%%%%%%%%%%%%%%%%%%%%%%%%%%%
% ADDITIONAL
\newcommand\tmbarlineloop[4][]{\foreach \tm@i in {#2} {\tmbarline[#1]{\tm@i}{#3}{#4}}}